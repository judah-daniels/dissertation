% Template for a Computer Science Tripos Part II project dissertation
\documentclass[12pt,a4paper,twoside,openright]{report}
\usepackage[pdfborder={0 0 0}]{hyperref}    % turns references into hyperlinks
\usepackage[left=25mm, right=25mm, bottom=25mm, top=20mm]{geometry}  % adjusts page layout
\usepackage{graphicx}  % allows inclusion of PDF, PNG and JPG images
\usepackage{verbatim}
\usepackage{amsfonts}
\usepackage{placeins}
\usepackage{amsthm}
\usepackage{amsmath}
\usepackage{tabu}
\usepackage{docmute}   % only needed to allow inclusion of proposal.tex
\usepackage[utf8]{inputenc}
\usepackage{mathtools}
\usepackage{changepage}
\usepackage{url}
\usepackage{blindtext}
\usepackage{tabularx,booktabs}
\usepackage{dirtree}
\usepackage{cite}
\usepackage{float} % Prevent Latex from repositioning tables
\usepackage{graphicx}

%--- Remove hbox warning
\hfuzz=5000.002pt 
%----------------------------------------------------------------------------------------

% Formatting Commands
\newcommand{\keyword}[1]{\textbf{#1}}
\newcommand{\tabhead}[1]{\textbf{#1}}
\newcommand{\code}[1]{\texttt{#1}}
\newcommand{\file}[1]{\texttt{\bfseries#1}}
\newcommand{\option}[1]{\texttt{\itshape#1}}

%----------------------------------------------------------------------------------------
%% *****************************************************************
\newlength{\upBranch} % shift up the text  lines <<<<
\setlength{\upBranch}{0.7ex} % 

\newlength{\tolineSpace} % blank space bellow text  lines  <<<
\setlength{\tolineSpace}{1mm}% 

\usepackage{xpatch} % needed <<<<<<<<
\makeatletter

\xpatchcmd{\dirtree} % root
{\vbox{\@nameuse{DT@body@1}}}
{\raisebox{-\tolineSpace}{\vbox{\@nameuse{DT@body@1}}}}
{}{}    

\xpatchcmd{\dirtree} % below space
{\advance\dimen\z@ by-\@nameuse{DT@lastlevel@\the\DT@countiv}\relax}
{\advance\dimen\z@ by-\tolineSpace \advance\dimen\z@ by-\@nameuse{DT@lastlevel@\the\DT@countiv}\relax}
{}{}
    
\xpatchcmd{\dirtree}% shift up the text  lines
{\kern\DT@sep\box\z@\endgraf}
{\kern\DT@sep\raisebox{-\upBranch}{\box\z@}\endgraf}
{}{}    

\makeatother
%% *****************************************************************


%-----
% Language setting
\usepackage[english]{babel}
%\raggedbottom                           % try to avoid widows and orphans
\sloppy
\clubpenalty1000%
\widowpenalty1000%

\renewcommand{\baselinestretch}{1.1}    % adjust line spacing to make
                                        % more readable


\theoremstyle{definition}
\newtheorem{definition}{Definition}[section]
\graphicspath { {figs/}}

\begin{document}

% Change these

\newcommand{\mcandidate}{2200D}
\newcommand{\mfullname}{Judah Daniels}
\newcommand{\mcollege}{Clare College}
\newcommand{\mtitle}{Inferring Harmony from Free Polyphony}
\newcommand{\ntitle}{Inferring Harmony from Free Polyphony}
\newcommand{\mexamination}{Computer Science Tripos -- Part II}
\newcommand{\mdate}{July, 2023}
\newcommand{\moriginator}{Christoph Finkensiep}
\newcommand{\msupervisor}{Dr Peter Harrison}
\newcommand{\mwordcount}{5434}
\newcommand{\mlinecount}{2272}
% Consent to the dissertation made available to University members
\newcommand{\mconsent}{I am content for my dissertation to be made available to the students and staff of the University.}
% For the Declaration of originality
\newcommand{\msignature}{Judah Daniels}


\bibliographystyle{acm}

%%%%%%%%%%%%%%%%%%%%%%%%%%%%%%%%%%%%%%%%%%%%%%%%%%%%%%%%%%%%%%%%%%%%%%%%
% Title
%%%%%%%%%%%%%%%%%%%%%%%%%%%%%%%%%%%%%%%%%%%%%%%%%%%%%%%%%%%%%%%%%%%%%%%%

\thispagestyle{empty}

\rightline{\LARGE \textbf{\mfullname}}

\vspace*{60mm}
\begin{center}
\Huge
\textbf{\mtitle} \\[5mm]
\mexamination \\[5mm]
\mcollege \\[5mm]
\mdate  % today's date
\end{center}

%%%%%%%%%%%%%%%%%%%%%%%%%%%%%%%%%%%%%%%%%%%%%%%%%%%%%%%%%%%%%%%%%%%%%%%%%%%%%%
% Proforma, table of contents and list of figures
%%%%%%%%%%%%%%%%%%%%%%%%%%%%%%%%%%%%%%%%%%%%%%%%%%%%%%%%%%%%%%%%%%%%%%%%%%%%%%

\pagestyle{plain}

\newpage
\newpage
\section*{Declaration of originality}

I, \mfullname{} of \mcollege, being a candidate for Part II of the Computer Science Tripos, hereby declare that this dissertation and the work described in it are my own work, unaided except as may be specified below, and that the dissertation does not contain material that has already been used to any substantial extent for a comparable purpose. \mconsent

\bigskip
\leftline{Signed \msignature}
\bigskip
\leftline{Date \today}

\chapter*{Proforma}


{\large
\begin{tabular}{ll}
Candidate Number:   & \bf \mcandidate                   \\
Project Title:      & \bf \mtitle                       \\
Examination:        & \bf \mexamination, \mdate         \\
Word Count:         & \bf \mwordcount\footnotemark[1]   \\
Code Line Count:    & \bf \mlinecount                   \\
Project Originator: & \bf \moriginator                  \\
Supervisor:         & \bf \msupervisor                  \\ 
\end{tabular}
}

\footnotetext[1]{This word count was computed
by \texttt{detex diss.tex | tr -cd '0-9A-Za-z $\tt\backslash$n' | wc -w}
}
\stepcounter{footnote}


\section*{Original Aims of the Project}
% At most 100 words



\section*{Work Completed}
% At most 100 words

All that has been completed appears in this dissertation.

\section*{Special Difficulties}
% At most 100 words

None

\newpage

\tableofcontents

\listoffigures

\newpage
\section*{Acknowledgements}



%%%%%%%%%%%%%%%%%%%%%%%%%%%%%%%%%%%%%%%%%%%%%%%%%%%%%%%%%%%%%%%%%%%%%%%
% now for the chapters

\pagestyle{headings}

\chapter{Introduction}
\textit{This dissertation explores efficient search strategies for parsing a symbolic music data using a musical grammar. We present ... which extends a recent model, addressing problems of intractability by using Heuristic search methods. We will see that that my novel heuristic search algorithm achieves .... }

\section{Motivation}

A piece of music can be described using a sequence of chords, representing a higher level harmonic structure of a piece. There is a small, finite set of chord types, but each chord can be realised on the musical surface in a practically infinite number of ways. Given a score, we wish to infer the underlying chord types. This is often a time consuming and cognitively demanding expert task \cite{masadaChordRecognitionSymbolic2018}, hence its utility.

\par
The paper \textit{Modeling and Inferring Proto-voice Structure in Free Polyphony} describes a generative model that encodes the recursive and hierarchical dependencies between notes, giving rise to a grammar-like hierarchical system \cite{finkensiepMODELINGINFERRINGPROTOVOICE2021}. This proto-voice model can be used to reduce a piece into a hierarchical structure which encodes an understanding of the tonal/harmonic relations.
\par
Finkensiep suggests in his thesis that the proto-voice model may be an effective way to infer higher level latent entities, such as harmonies or voice leading schemata. Thus in this project I will ask the question: is this parsing model an effective way to annotate harmonies? By ‘effective’ we are referring to two things:
\begin{itemize}
  \item Accuracy: can the model successfully emulate how experts annotate harmonic progressions in musical passages? 
  \item Practicality: can the model be used to do this within a reasonable time frame?
\end{itemize}

While the original model could in theory be used to generate harmonic annotations, its exhaustive search strategy would be prohibitively time-consuming in practice for any but the shortest musical extracts; one half measure can have over 100,000 valid derivations \cite{finkensiepStructureFreePolyphony2023}. My approach will be to explore the use of heuristic search algorithms to solve this problem.

\section{Related Work}
First grammar/rule-based systems \cite{maxwellExpertSystemHarmonizing1992} \cite{winogradLinguisticsComputerAnalysis1968}, optimisation algorithms \cite{pardoAlgorithmsChordalAnalysis2002} and supervised learning approaches making use of large datasets \cite{niEndtoendMachineLearning2011} \cite{mcleodModularSystemHarmonic2021} \cite{masadaChordRecognitionSymbolic2018}.
\par
Distinction to be drawn between approaches that take segmented or non-segmented pieces. Some models make use of a joint segmentation and labelling approach \cite{masadaChordRecognitionSymbolic2018}... 

\par
More general heuristic search techniques for huge search spaces.

%%%%%%%%%%%%%%%%%%%%%%%%%%%%%%%%%%%%%%%%%%%%%%%%%%%%%%%%%%%%%%%%%%%%%
% Preparation
%%%%%%%%%%%%%%%%%%%%%%%%%%%%%%%%%%%%%%%%%%%%%%%%%%%%%%%%%%%%%%%%%%%%%

\chapter{Preparation}
\textit{In this chapter, I present the work which was undertaken before the code was written. After a brief description of my starting point, I provide an exposition of the Proto-voice Model which forms the foundation of this project. Subsequently, I discuss probabilistic programming and Bayesian inference, including a probabilistic model of harmony. Finally, I describe the software engineering techniques and principles used throughout the project. }

\section{Background Material}

\subsection{Proto-voice Model}
TODO: Brief description of what notes are, what is a piece of music, introduce relevant musical terminology (Score, note, protovoice, repetition, neighbor/ ornament(choose one?)). Conflict of meaning with root-note (generative operation) and root note (musical terminology). What are the main assumptions/ information we need to know in order to understand the proto-voice model?

\par
TODO: Motivation of the model as a generative model on the note level, describing the piece as a DAG (Directed Acyclic Graph). What is a proto-voice exactly? 
\par 

The proto-voice model is characterised by 3 primitive generative operations on notes.

\begin{itemize}
  % \setlength\itemsep{1em}
 \item Repetitions 
  \item Neighbor notes 
  \item Passing notes
\end{itemize}

Operations with two notes are represented by edge replacement. 
\[p_1 \to p_2 ~~~\implies~~ p_1 \to c \to p_2 \label{edge replacement}\]

\subsubsection{Proto-voice Operations}

To model simulataneity of notes we introduce slices, which are multisets of pitches, representing segments of a piece where a group of notes are heard. 
\par 
Diagram showing a slice + a diagram showing a higher level slices, grouping an arpegiation.
\par 
A slice $m$ is defined as a multiset of pitches.
\par 
A transition $t = (s_l, e, s_r)$  relates two slices with a configuration of edges $e=(e_{reg}, e_{pass})$, a set of regular edges (repetition or neighbor), and a set of passing edges.
\par
Outer operations (Diagram of all three operations): 
\par
Split: \[t \to t'_l s' t'_r\]
Spread: \[t_l~s~t_r \to t'_l~s'_l~t'_m~s'_r~t'_r\]
Freeze: \[t \to t \]

\subsubsection{Proto-voice harmony}
How do we get from a proto-voice (partial)derivation to a harmonic inference?
\par
Explain what harmony is, and how the proto-voice model allows us to capture harmony. Elaboration of the introduction.
\par
What assumptions are needed for a protovoice derivation to be able to describe harmonic entities? These shape heuristic design.
\par


\subsection{Probabilistic Programming}
Provide an explanation of all the concepts I learned and used in this project. 
Techniques such as marginalisation, joint distributions, bayes rule etc. 
Probabilistic programming is the combination model definitions and statistical inference algorithms for computing the conditional distribution of inputs (chords) that could have given rise to the observed output (score). We are making the assumption that the score is a realisation of the latent harmonic entities. 

Dirchelet distributions
Beta distribution  
Multinomiall distribution 
Normal Distribution 

\par 
Inference as model $\to$ data $\to$ prob distribution $\to$ chord guess

\subsection{Probabilistic Model of Harmony}

Outline of the probabilistic model of harmony, describing the parts that are relevant for harmonic annotations. This section allows the reader to understand the evaluation and heuristic modules.

\par
Describe how the parameters were attained
\par
Then describe how to go from the parameters to chord, chordtone and ornamentation distributions
\par

Parameters:
\par
pHarmonies: $\mathbb{N}^{n_c}$\\
pChordtones: $\mathbb{N}^{n_c}$\\
pOrnaments: $\mathbb{N}^{n_c}$
\par
Chordtypes, $C = \{\text{M,~m, Mm7, om, o7, mm7, \%7, MM7, +, Ger, It, Fr, mM7, +7}\}$

\[\vec{\chi}' \sim \text{Dirchlet} (\text{pHarmonies}, n_c) \]

\[\vec{\chi} = \mathbb{E} (\vec{X}_i) = \frac{\alpha_i}{\sum\limits_j \alpha_j} \]

Chord: \[c \sim \text{Categorical}(\vec{\chi})\]

Single chordtone distribution. We want to find $P(p|c, ct)$ probability of the pitch given the chord, and that the note is a chordtone:
\[\vec{\phi}_{ct}' \sim \text{Dirchlet}(pChordtones, n_p) \implies \vec{\phi}\]

For each of these parameters we use the MLE to get our probability distribution. 
\[\vec{\phi}_{ct} = \text{MLE} (\vec{\phi}_{ct}')\]
\[\vec{\phi}_{or} = \text{MLE} (\vec{\phi}_{or}')\]
\[\vec{\chi}= \text{MLE} (\vec{\chi}') \]
Then for each chord tone,
\[p_{ct} \sim \text{Categorical}(\vec{\phi}_{ct})\]
\[p_{or} \sim \text{Categorical}(\vec{\phi}_{or})\]
We get the distribution of likelihoods for each pitch.






\subsection{Heuristic Search Algorithms}

Provide an outline of the heuristic search paradigm with a formalisation.
\par
Provide a brief overview of different techniques that are used to prune the search space that might be relevant.

\section{Starting Point}

\subsection{Relevant courses and experience}

\paragraph{Haskell}{I was introduced to Haskell during an internship during the summer before starting this project (July to August 2022). This project is an excuse to learn the language.}
\paragraph{Python}{I have experience coding in Python.}

\paragraph{IB}{ Formal Models of Language, Artificial Intelligence.}

\subsection{Existing codebase}
NOTE: I need to restructure the codebase to reflect the different modules more clearly. This will help with the flow to the dissertation.

The following describes the protovoices-haskell repository, and where my code contribution will lie:
\par
\medskip
\dirtree{%
.1 protovoices-haskell.
.2 app.
.3 MainExamples.hs.
.3 MainISMIR.hs.
% .3 MainLearning.hs.
% .3 MainHeuristicSearch.hs <- My code.
.3 ....
.2 src.
.3 \textbf{Heuristics} <- My code.  
% .4 …  <- My code.
.3 ....
.2 test.
% .2 testdata.
.2 ....
}

\section{Requirements Analysis}

Table of main components with a dependency and risk analysis 

\begin{table}[ht]
  \caption{Overview of main deliverables along with a risk analysis}
  \label{requirements}
  \begin{tabularx}{\textwidth}{cXcc}
    ID & Delieverable & Priority & Risk \\
    \toprule
    \texttt{core1} & Evaluation Module & High & Low \\
    \texttt{core2} & End to End Pipeline  & High & Medium \\
    \texttt{core3} & Parser  & High & Medium \\
    \texttt{base1} & Random Choice search & High & Low \\
    \texttt{base2} & Random Sample & High & Low \\
    \texttt{ext1} & Heuristic Search 1 & Medium & High \\
    \texttt{ext2} & Heuristic Search 2 & Medium & High \\
  \end{tabularx}
\end{table}
\par
Short description of where the risk lies. 
\par
Pertt chart showing the dependencies between different modules


\section{Software Engineering Techniques}
Justified and documented selection of suitable tools; good engineering approach.


\subsection{Development model}

Usig the dependency and risk analysis above, I created this gantt chart, and totally stuck to it(100\% didn't wait until now to get the end-to-end pipeline fully running, and spend most of the time in an extension rabbit hole.. We live and we learn).

Include Gantt chart.

\subsection{Languages, libraries and tools}
The chapter will also cite any new programming languages and systems which had to be learnt 

\begin{table}
  {
  \small
  \caption{Languages, libraries and tools}
  \label{Languages}
  \begin{center}
    \begin{tabularx}{.9\textwidth}{cXc}
      Tool & Purpose & License \\
      \toprule
      Haskell & Main language & ... \\
      \midrule
      GHC & Compiling and profiling to inspect time performance and memory usage  & GPL-3.0+ \\
      \midrule
      Haskell-Musicology & ... & ... \\
      \midrule
      Dimcat & ... & ... \\
      \midrule
      Python & ... & ... \\
      \midrule
      Numpy & ... & ... \\
      \midrule
      Pandas & ... & ... \\
      \midrule
      MS3 & ... & ... \\
      \midrule
      Musescore 3 & ... & ... \\
      \midrule
      Protovoice Annotation Tool & ... & ... \\
      \midrule
      Git & Version Control, Continuous Integration & ... \\
      \bottomrule
    \end{tabularx}
  \end{center}
  }
\end{table}

%%%%%%%%%%%%%%%%%%%%%%%%%%%%%%%%%%%%%%%%%%%%%%%%%%%%%%%%%%%%%%%%%%%%%
% Implementation
%%%%%%%%%%%%%%%%%%%%%%%%%%%%%%%%%%%%%%%%%%%%%%%%%%%%%%%%%%%%%%%%%%%%%

\chapter{Implementation}

\section{Repository Overview:}

Insert block diagram of components here.

\vspace{50\baselineskip}

\DTsetlength{0em}{1.3em}{0em}{0.7pt}{3pt}       
\setlength{\DTbaselineskip}{15pt}  %minimum size for \normalsize
\renewcommand{\DTstyle}{\ttfamily}

\begin{table}[!t]
  % \centering
  \caption{Repository Overview}
  \vspace{\baselineskip}
  \label{jeff}
  \begin{tabularx}{\textwidth}{l X c}
    File/Folder & Description & LOC \\
    \toprule
    \toprule
  \begin{minipage}[t]{5.3cm}
    \dirtree{%
    .1 protovoices-haskell/.
    .2 src/.
    .3 HeuristicParser.hs,~HeuristicSearch.hs \vspace{\DTbaselineskip}.
    .3 RandomChoiceSearch.hs,~RandomSampleParser.hs\vspace{2\DTbaselineskip}.
    .3 Heuristics.hs,~PBHModel.hs \vspace{2\DTbaselineskip}.
    .3 FileHandling.hs\vspace{2\DTbaselineskip}.
    .3 \dots \vspace{\DTbaselineskip}.
    .2 app/.
    .3 MainFullParse.hs\vspace{\DTbaselineskip}. 
    .2 harmonic-inference \vspace{\DTbaselineskip}.
    .2 experiments/.
    .3 preprocess.ipynb.
    .3 dcml\_params.json.
    .3 inputs/ \vspace{\DTbaselineskip}.
    .2 test/ \vspace{\DTbaselineskip}.
    }
  \end{minipage} &
  \begin{minipage}[t]{8cm}
Root directory
\vspace{2\baselineskip}\\
Core Implementation (Section x)
\vspace{2\baselineskip}\\
Baseline Implemetation (Section x)
\vspace{2\DTbaselineskip}\\
Extension Implementation (Section x) 
\vspace{2\DTbaselineskip}\\
Utilities
\vspace{5\DTbaselineskip}\\
Entry Point
\vspace{8.2\DTbaselineskip}\\
Unit Tests (Section x)

  \end{minipage} & 
  \begin{minipage}[t]{0.5cm}
    2272
    \vspace{0.1\DTbaselineskip}\\
    470\\
    \vspace{\DTbaselineskip}
    121\\
    \vspace{\DTbaselineskip}
    383\\
    \vspace{1.8\DTbaselineskip}
    188\\
    \vspace{3.7\DTbaselineskip}
    431\\
    \vspace{3\DTbaselineskip}
    115\\
    \vspace{2.5\DTbaselineskip}
    611\\
  \end{minipage}
\end{tabularx}
\end{table}

The following describes an overview of the project repository:

\section{Core Implementation}

\subsection{Heuristic Parser}
This is not a descriptive name. Think of a new name to describe the implementation of the search space of partial reductions. We use the outer representation of structure and outer operations. This is an abstraction.

\subsubsection{Parsing Operations}
Piece represented by an alternating list of slices and transitions, this is called a path. Define path formally. inductive definitions. dont need the Nothing: just Path trans slice. Transition can be frozen or unfrozen, and boundary or non boundary. Boundary is represented by vertical line, frozen is represented by two lines.

\begin{figure}[ht]
  \centering
  \includegraphics[width=\textwidth]{pathFromSlices}
  \caption{Path initiliasation}
  \label{fig:pathInit}
\end{figure}

\begin{definition}[Path]
A path is an alternating sequence of an two types of elements, in our case transitions and slices. Definition: Haskell code block or mathematical definition?
\end{definition}

Our goal is to reduce the piece into a partial redution by appluying operations until we have one slice per segment. Diagram of this state. This means we have one group of notes per segment, and this group of notes should represent the harmony of the segment.

We parse by applying the inverse of the generative operations, right to left. Unsplit, Unspread, Unfreeze.
\par


\begin{figure}[h]
  \centering
  \includegraphics[width=\textwidth]{parseops}
  \caption{Parse operations}
  \label{fig:parseops}
\end{figure}

\FloatBarrier
\subsubsection{State Space}

This is how we define the search. We start at the right, the end of the piece. We have a pointer to the current node, and all preceeding slices are open and subsequent slices are frozen. Open Slices can be reduced, but only to the point that there is one slice in a segment. We keep track of the operations performed as it (1). allows us to the draw out the derivation for the partial reduction at the end, and (2). it is used later for calculate a cost for each operation for the heuristic search.




% \begin{figure}[h]
%   \centering
%   \includegraphics[width=\textwidth]{parsestates}
%   \caption{Parse operations}
%   \label{fig:parseops}
% \end{figure}

\begin{figure}[h]
  \centering
  \includegraphics[width=\textwidth]{searchstate}
  \caption{Search state}
  \label{fig:searchstate}
\end{figure}
\par
\par

\FloatBarrier
\subsubsection{Enumerating State transitions}

\begin{figure}[h]
  \centering
  \includegraphics[width=0.7\textwidth]{frozenenum}
  \caption{Unfreeze operation}
  \label{fig:frozenenum}
\end{figure}


\begin{figure}[h]
  \centering
  \includegraphics[width=0.7\textwidth]{sssemiopenenum}
  \caption{Enumeration of operations mid parse. Maybe for appendix? This could be much more concise.}
  \label{fig:sssemiopenenum}
\end{figure}

\begin{figure}[h]
  \centering
  \includegraphics[width=\textwidth]{statetransitions}
  \caption{State Transition diagram}
  \label{fig:statetrans}
\end{figure}
\FloatBarrier

In the state transition diagram (Figure \ref{fig:statetrans}), we see all the possible parse states (Is this actually useful? Maybe for appendix). This was useful for me as it helps to conceptualise how the full parse actually works. The dimensions of this digramm of the search state depends on the length of the piece, and the size of each segment. We can see that there is a process of moving to the right to unfreeze transitions, and moving towards the left during reduction operations. Perhaps some simplification of the diagram would be useful. This transition diagram does not consider segment boundaries.

sdfsdfs

sdf

\FloatBarrier
\subsubsection{Boundary handling}

It is important thhat we don't reduce to an empty segment, because that would mean we've lost all information about the segment, and would not be able to make a harmonic inference. In order to prohibit this, we add additional constraints to the parse operations for each opertation based on the boolean boundary value of all involved transitions.
\par
We use karnaugh maps to determine the boolean expression for these constraints.

\begin{figure}[h]
  \centering
  \includegraphics[width=0.7\textwidth]{karnaughunfreeze}
  \caption{Determine boolean boundary expressions for the freeze operations}
  \label{fig:karnaugh}
\end{figure}

\par 
Could show other maps in the appendix.


\FloatBarrier
\subsection{Evaluation Module}
We need to know exactly what we are trying to achieve before we can understand the baseline and extention implementations.

\FloatBarrier
\subsubsection{Probabilistic Model of Harmony}
When evaluating using the protovoice model: we assume that we result in only chord tones for each segment. Thus we can use the chord tone probabilities to evaluate the prediction. 
\par
When just using a random sample, we have to assume that there is a mixture model of chord tones and ornaments. We can use the learnt parameters to determine the distribution.
\par
These two measures of likelihoods are comparable as they are drawn from the same distributions.
\par
We also need to infer chord labels. We can simply choose the chord that is most likely according to our model.
\par 
This gives us two key metrics, likelihood and accuracy.
\par
Could also use a more sophisticated notion of accuracy, using a chord similarity function \cite{humphreyFourTimelyInsights2015}. The {\texttt {mir\_eval}} package provides a plethora of metrics to compare chord label predictions \cite{raffelMirEvalTransparent2014}. 

\section{Baseline implementation}


\subsection{Random Sample Parser}
As a crude baseline we develop two algorithms based on randomly sampling notes for each segment to infer the chord label. 
\par
The pure random sample algorithm simply samples random notes for each segment, and uses those to guess the chord label. This doesn't even consider the notes of the piece, so it's really bad, but provides a useful reference.
\par 
The per segment sample algorithm samples notes from each segment. Could just sample a random number of notes from each segment, or just use all the notes in the segment to predict the most likely chord label. This is reminiscent of using a key-profile model \cite{temperleyBayesianApproachKeyFinding2002} to find local keys.

\FloatBarrier
\subsection{Random Choice Search}
Now we use our implementation of the protovoice parser, but just do a random walk in the tree of partial reductions. By comparing this against the random ample parser, we can get an idea of the utility of the model. We show that this works surprisingly well.

\FloatBarrier
\section{Extension Implementation}

\subsection{Heuristic Design}
Step 1: Design heuristic to be as accurate as possible. I.e the extreme is to consider every possible parse, but for a single piece there can be over $10^{10^{10^{10^{10^{10}}}}}$ different parses. We consider 1 step at a time at first - this still results in needing to choose an operation out of upwards of 30,000,000 options for just a single step.     
\par
First the full piece heuristic parse 
\par
Problem of very large slices.\\ 
Segment by segment heuristic parse - avoids the problem, but is slightly hacky. Can we incorprate our knowledge regarding the relative proportion of chord tones and ornaments. Should we allow duplicates of notes in slices? Perhaps we should favour spreads more. 
\par
Always consider a certain number of slices and spreads.
\par

\begin{figure}[h]
  \centering
  \includegraphics[width=0.7\textwidth]{splitsssemiopen}
  \caption{Split operation}
  \label{fig:splitoperation}
\end{figure}

\FloatBarrier
\subsubsection{Scoring Unsplit Operations}
Consider the Split rule : \[t \to t'_{l}~s'~t'_{r}\]
\par
During a split, each edge in the transition and each node in an adjacent slice can be elaborated by one or more inner operations.
These new edges can be discarded or kept to form the new edge of $t'_l$ and $t'_r$.
\par 
The notes in the child slice $s$ can either have edges connected to the left neighboring slice or right neighbouring slice, or both. I.e for each note in the child slice, it can be a an ornmentation of a previous note, subsequent note, both, or repetition of prev note, subsequent note etc. So we consider the chord tone profiles of the involved slices. 

We first guess the chord type each parent slice. 
\[\theta_l = \mathop{argmax}_{c \in C} P(s_l|c) ~~,~~ \theta_r = \mathop{argmax}_{c \in C} P(s_r|c) '\]

We now consider each edge individually, considering their likelihoods based on the proabilistic model of harmony along with theoretical assumptions. 

\paragraph{Single Sided Operations} 
\begin{itemize}
  \item Right Neighbour (Left Neighbour anagolously)
    \[ x \implies x \to n~~, x,n \in P \]
    \[x \sim \text{Categorical}(\sigma_{ct}^{\theta_l})\]
    \[n \sim \text{Categorical}(\sigma_{or}^{\theta_r})\]
    Find \[P(x,n~|~\theta_l)\]
  \item Right Repeat (Left Repeat anagolously)
    \[ x \implies x \to x~~, x \in P \]
    \[x \sim \text{Categorical}(\sigma_{ct}^{\theta_l})\]
    Find \[P(x~|~\theta_l)\]
\end{itemize}
\paragraph{Two Sided Operations} 
\begin{itemize}
  \item Root Note: This operation is only done once in the original model. In our case we do not need to consider due to segment boundaries.
  \item Full Repeat: 
    \[ x \implies x \to n~~, x,n \in P \]
    \[x \sim \text{Categorical}(\sigma_{ct}^{\theta_l})\]
    \[n \sim \text{Categorical}(\sigma_{or}^{\theta_r})\]
    Find \[P(x,n~|~\theta_l)\]
  \item Left Repeat of Right: 
    \[ x \to y \implies x \to y' \to y \]
    \[y \sim \text{Categorical}(\sigma_{ct}^{\theta_l})\]
    Find \[P(y~|~\theta_l)\]
  \item Full Neighbour:
    \[ x_1 \to x_2 \implies x_1 \to n \to x_2, x \in P \]
    % \[x \sim \text{Categorical}(\sigma_{ct}^{\theta_l})\]
    Find \[P(|~\theta_l,\theta_r)\]
\end{itemize}

\FloatBarrier
\subsubsection{Scoring Unspread Operations}

Consider the Spread rule : \[t_l s_r \to t'_l s_l t'_m s_r t'_r\]
We make the assumption that $s$, $s_l$, \& $s_r$ are all realisations of the same chord. This lines up with the music theorretical basis for this operation in the model(justify).
\par 
Thus we find the most likely chord (optional extension: marginalise over all chords)
\[\theta = \mathop{argmax}_{c \in C} P (s|c)\]
\par 
When then measure the extent to which the parent slics match this chord.

\[p(s_l, s_r| \theta)\]

We can calculate $p(s_l|\theta)$ and $p(s_r|\theta)$ using the multinomial distribution probability density function as described in the preparation chapter.

\FloatBarrier
\subsubsection{Scoring Unfreeze Operations}
We assign 0 cost to unfreeze operations. This means we need to be careful about ensure that we don't just unfreeze the entire piece immediately. Careful construction of the search algorithm can ensure this. More later.

\FloatBarrier
\subsubsection{Full state evalutation}
We need to combine all of these in a fair way. Also the distinction between splits and spreads need to be considred, as they are different operations, the calculations of likelihood may cause an imbalance. 

\FloatBarrier
\subsection{Heuristic Search}
Step 2: Relax the heuristic search in order to reduce runtime/ lower complexity.
\par 
In the case that there are 85,000,000 options, perhaps we should sample the options rather than evaluating all of them. 
\par 
This version of heuristic search should be able to parse full pieces (hopefully), so can be used to compare with the baselines on an entire corpus.



\section{Testing}
Show unit tests, and examples of the test/development cycle for the heuristic search development

%%%%%%%%%%%%%%%%%%%%%%%%%%%%%%%%%%%%%%%%%%%%%%%%%%%%%%%%%%%%%%%%%%%%%
% Evaluation
%%%%%%%%%%%%%%%%%%%%%%%%%%%%%%%%%%%%%%%%%%%%%%%%%%%%%%%%%%%%%%%%%%%%%

\chapter{Evaluation}
\textit{In this chapter, I provide qualitative and quantitative evaluations of the work completed. I then provide and interpret evidence to show that the success criteria were met.}

\textit{The main questions to answer are as follows:}
\begin{itemize}
  \item \textit{Can the proto-voice model be used to accurately infer chord labels?}
  \item \textit{Can the proto-voice model be used to practically infer chord labels?}
  \item \textit{How well my heuristic search algorithms infer chord labels?}
\end{itemize}

\section{Accuracy}
Things to note
\begin{itemize}
  \item The fact that segmentation is known ahead of time provides a great deal of information \cite{gothamWhatIfWhen2021}
  \item So we can use comparisons between the random sample from each segment algorithm and the random parse algorithm to see if the use of the grammar provides an advantage over just sampling the notes directly, without looking at relations between notes.
  \item Then we want a heuristic search algorithm that considers each option exhaustively and finds the best local option. This is too computationally expensive to be used for whole pieces. 
  \item Given there can be millions of possible next states in the search, we need to look at different strategies to avoid searching through them all. E.g just sample states. 
  \item Sensitivity Analysis for the heuristic search is useful for the evaluation. Explore how robust it is to handcrafted attacks/ different types of passages.
  \item Could evaluate by segments instead of pieces. 
\end{itemize}

\section{Performance}

\section{Heuristic Search (Extension)}

\section{Success Criteria}

\section{Limitations}


%%%%%%%%%%%%%%%%%%%%%%%%%%%%%%%%%%%%%%%%%%%%%%%%%%%%%%%%%%%%%%%%%%%%%
% Conclusions
%%%%%%%%%%%%%%%%%%%%%%%%%%%%%%%%%%%%%%%%%%%%%%%%%%%%%%%%%%%%%%%%%%%%%

\chapter{Conclusions}
\textit{In this chapter, I first discuss the success achieved by the project then offer a reflection on lessons learned. Finally, I consider the directions in which there is potential for future work.}
\section{Achievements}

\section{Lessons learned}

\section{Future Work}

\cite{finkensiep_modeling_2021}


%%%%%%%%%%%%%%%%%%%%%%%%%%%%%%%%%%%%%%%%%%%%%%%%%%%%%%%%%%%%%%%%%%%%%
% the bibliography
%%%%%%%%%%%%%%%%%%%%%%%%%%%%%%%%%%%%%%%%%%%%%%%%%%%%%%%%%%%%%%%%%%%%%
\addcontentsline{toc}{chapter}{Bibliography}
\nocite{*}
% \addbibresource{Disseration.bib}
\bibliography{Dissertation}


%%%%%%%%%%%%%%%%%%%%%%%%%%%%%%%%%%%%%%%%%%%%%%%%%%%%%%%%%%%%%%%%%%%%%
% the appendices
%%%%%%%%%%%%%%%%%%%%%%%%%%%%%%%%%%%%%%%%%%%%%%%%%%%%%%%%%%%%%%%%%%%%%

\appendix

\chapter{Additional Information}

% \section{metadata.tex}
% {\scriptsize\verbatiminput{metadata.tex}}
%
% \section{main.tex}
% {\scriptsize\verbatiminput{main.tex}}
%
% \section{proposal.tex}
% {\scriptsize\verbatiminput{proposal.tex}}
%
% \chapter{Makefile}
%
% \section{makefile}\label{makefile}
% {\scriptsize\verbatiminput{makefile.txt}}
%
% \section{refs.bib}
% {\scriptsize\verbatiminput{refs.bib}}


\chapter{Project Proposal}

\documentclass{article}

% Packages
\usepackage{tabularx,booktabs}
\usepackage{dirtree}
\usepackage{cite}
\usepackage{float} % Prevent Latex from repositioning tables
\usepackage{amsmath}
\usepackage{graphicx}
\usepackage[colorlinks=true, allcolors=blue]{hyperref}

%----------------------------------------------------------------------------------------

% Formatting Commands
\newcommand{\keyword}[1]{\textbf{#1}}
\newcommand{\tabhead}[1]{\textbf{#1}}
\newcommand{\code}[1]{\texttt{#1}}
\newcommand{\file}[1]{\texttt{\bfseries#1}}
\newcommand{\option}[1]{\texttt{\itshape#1}}

%----------------------------------------------------------------------------------------

% Language setting
\usepackage[english]{babel}

% Set page size and margins
% Replace `letterpaper' with `a4paper' for UK/EU standard size
\usepackage[a4paper,top=2cm,bottom=2cm,left=3cm,right=3cm,marginparwidth=1.75cm]{geometry}

\begin{document}

\title{Inferring Harmony from Free Polyphony}
\author{Judah Daniels}
\date{\parbox{\linewidth}{\centering%
  \today\endgraf\bigskip
  DOS: Prof. Larry Paulson\endgraf\medskip
  Crsid: jasd6 \endgraf
  College: Clare College}}
\maketitle


\section{Abstract}
% Further Explanation of the background and the objectives.
A piece of music can be described using a sequence of chords, representing a higher level harmonic structure of a piece. There is a small, finite set of chord types, but each chord can be realised on the musical surface in a practically infinite number of ways. Given a score, we wish to infer the underlying chord types. 
\par
The paper \textit{Modeling and Inferring Proto-voice Structure in Free Polyphony} describes a generative model that encodes the recursive and hierarchical dependencies between notes, giving rise to a grammar-like hierarchical system \cite{finkensiep_modeling_2021}. This proto-voice model can be used to reduce a piece into a hierarchical structure which encodes an understanding of the tonal/harmonic relations of a piece. 
%A reduction is represented by a sequence of slices, where each slice is a set of pitches with a pointer to a starting and ending surface position
\par
Christoph Finkensiep suggests in his paper that the proto-voice model may be an effective way to infer higher level latent entities, such as harmonies or voice leading schemata. Thus in this project I will ask the question: is this parsing model an effective way to annotate harmonies? By `effective' we are referring to two things: \begin{itemize}
  \item Accuracy: can the model successfully emulate how experts annotate harmonic progressions in musical passages? 
  \item Practicality: can the model be used to do this within a reasonable time frame?
\end{itemize}
% Harmonic annotations segment a piece, marking each segment with a chord label. The goal is then to obtain these labels from the score given the segmentation.
While the original model could in theory be used to generate harmonic annotations, its exhaustive search strategy would be prohibitively time-consuming in practice for any but the shortest musical extracts; one half measure can have over 100,000 valid derivations \cite{finkensiep_modeling_2021}. My approach will be to explore the use of heuristic search algorithms to solve this problem.
% My project will extend the use of this model into field of harmonic analysis by inferring harmonic annotations for a piece. 
% \par
% As the protovoice model is generative, this harmonic analysis will consist of applying the inverse of the generative operations in order to find a reduction. The search-space for this model is intractable; one half measure can have over 100,000 valid derivations\cite{finkensiep_modeling_2021}.
% The starting point is a parser provided with the paper which exhaustively searches for a solution, leading to an exponential blowup. My approach will be to use heuristic search algorithms to find a partial derivation which is sufficient to infer the underlying harmony.
% the goal is basically 
% just to find a reduction of each segment to a single slice, where 
% ideally (a) this slice reflects the chord label and (b) the reduction 
% itself is a "plausible" derivation (

%The general approach would be based on (heuristic) search. The search 
% space would be partial reductions, the starting point is the (unreduced 
% and segmented) surface, and the goal would be any reduction of the 
% segments to single slices. The connections between different states are 
% the reduction steps defined by the PV model.

% My project will involve development of a parser(s) that can parse pieces of music using the grammar described in the paper, and formulate a method to search within the space of possible parse trees for a suitable structural interpretation of a piece of music. This will be used to intelligently infer chord annotations given an arbitrary score.

% The dissertation includes code for a random parser which generates a derivation of the piece according to the protovoice model, by applying the generative process backwards, randomly choosing operations. 
%
% The specific problem of inferring harmonic annotations is not addressed in the dissertation, so theres is an open question of how one can go from a parse tree to a set of chord labels.
%

\section{Substance and Structure}
% Give details of specific goals to be achieved
% Give precise characterisations of the methods that will be used
% Data structures and algorithms
% Key concepts, major work items, their relations and relative importance
% Specify what it means for the project to be a success.
% Identify the main sub tasks
% Outline the main algorithms or techniques to be adopted in completed them
\subsection{Core: Search}
The core of this project is essentially a search problem characterised as follows:
\begin{itemize}
  \item The state space $S$ is the set of all possible partial reductions of a piece along with each reduction step that has been done so far. 
  \item We have an initial state $s_o \in S$, which is the empty reduction, corresponding to the unreduced surface of the piece. The score is represented as a sequence of slices grouping notes that sound simultaneously. We are also given the segmentation of the original chord labels that we wish to retrieve.
  \item We have a set of actions, $A$ modelled by a function $action: A \times S \to S$. These actions correspond to a single reduction step.
    \begin{itemize}
      \item The reduction steps are the inverses of the operations defined by the generative proto-voice model.
    \end{itemize}
  \item Finally we have a goal test, $goal: S \to \{true,false\}$ which is true iff the partial reduction $s$ has exactly one slice per segment of the input.
    \begin {itemize}
  \item This means the partial reduction $s$ contains a sequence of slices which start and end positions corresponding to the segmentation of the piece.
    \end {itemize}
  \item At the first stage, this will be implemented using a random graph search algorithm, picking each action randomly, according to precomputed distributions.
\end{itemize}
\par

\subsection{Core: Evaluation}

The second core task is to create an evaluation module that iterates over the test dataset, and evaluates the partial reduction computed by the search algorithm above. This will be done by comparing the outputs to ground truth annotations from the Annotated Beethoven Corpus.
\par
In order to do this I will make use of the statistical harmony model from Finkensiep's thesis, \textit{The Structure of Free Polyphony} \cite{finkensiep_structure_2022}. This model provides a way of mapping between the slices that the algorithm generates and the chords in the ground truth. This can be used to empirically measure how closely the slices match the expert annotations.   
\par

% from above by measuring the likelihood that the chord label produced the slice notes.

\subsection{Extension}
Once the base search implementation and evaluation module have been completed, the search problem will be tackled by heuristic search methods, with different heuristics to be trialled and evaluated against each other. The heuristics will make use of the chord profiles from Finkensiep's statistical harmony model discussed above. These profiles relate note choices to the underlying harmony. Hence the heuristics may include:
\begin{itemize}
  \item How the chord types relate to the pitches used.
  \item How the chord types relate which notes are used as ornamentation, and the degree of ornamentation.
  \item Contextual information about neighboring slices
\end{itemize}

\subsection{Overview}
\noindent
The main work packages are as follows:
\par
\medskip
\keyword{Preliminary Reading} -- Familiarise myself with the proto-voice model, and read up on similar models and their implementations. Study heuristic search algorithms.
\par
\medskip
\keyword{Dataset Preparation} -- Pre-process the Annotated Beethoven Corpus into a suitable representation for my algorithm.
\par
\medskip
\keyword{Basic Search} -- Implement a basic random search algorithm that takes in surface and segmentations, and outputting the sequence of slices matching the segmentations.
\par
\medskip
\keyword{Evaluation Module} -- Implement an evaluation module to evaluate the output from the search algorithm.
% Chord Profiles for chord tones and ornaments
% Existing set of probabilities: refer to the thesis.
\par
\medskip
\keyword{End-to-end pipeline} -- Implement a full pipeline from the data to the evaluation that can be used to compare different reductions.
\par
\medskip
\keyword{Heuristic Design} -- Extension -- Trial different heuristics and evaluate their performance against each other.
\par
\medskip
\keyword{Dissertation} -- I intend to work on the dissertation throughout the duration of the project. I will then focus on completing and polishing the project upon completion.

\section{Starting Point}
% Record any signifcant bodies of coe that will form a basis for your project that already exists
% Describe state of existing software
The following describes existing code and languages that will be used for this project:
\par
\bigskip
\keyword{Haskell} -- I will be using Haskell for this project as it is used in the proto-voice implementation. It must be noted that my experience with Haskell is limited, as I was first introduced to it via an internship this summer (July to August 2022).
\par
\bigskip
\keyword{Python} -- Python will be used for data handling. I have experience coding in Python.
\par
\bigskip
\keyword{Prior Research} - Over the summer I have been reading the literature on computational models of music, as well as various parsing algorithms such as semi-ring parsing \cite{goodman_semiring_1999}, and the CYK algorithm, which is used in the implementation of the proto-voice model.
\par
\bigskip
\keyword{Protovoices-Haskell} -- The paper \textit{Modeling and Inferring Proto-Voice Structure in Free Polyphony}\cite{finkensiep_modeling_2021} includes an implementation of the proto-voice model in Haskell. A fork of this repository will form the basis of my project.
This repository includes as parsing module which will be used to perform the actions in the search space of partial reductions. There is module that can exhaustively enumerate reductions of a piece, but this is infeasible in practice due to the blowup of the derivation forest.
\par
\bigskip
\keyword{MS3} -- This is a library for parsing MuseScore Files and manipulating labels\cite{johannes_ms3_2021}, which I will use as part of the data processing pipeline.
\par
\bigskip
\keyword{ABC} -- The \textit{Annotated Beethoven Corpus}\cite{neuwirth_annotated_2018} contains analyses of all Beethoven string quartets composed between 1800 and 1826), encoded in a human and machine readable format. This will be used as a dataset for this project. 
\par
%
% The following describes the protovoices-haskell repository, and where my code contribution will lie:
% \par
% \medskip
% \dirtree{%
% .1 protovoices-haskell.
% .2 app.
% .3 MainExamples.hs.
% .3 MainISMIR.hs.
% .3 MainLearning.hs.
% .3 MainHeuristicSearch.hs <- My code.
% .3 ....
% .2 src.
% .3 \textbf{Heuristics} <- My code.  
% .4 …  <- My code.
% .3 ....
% .2 test.
% .2 testdata.
% .2 ....
% }
\section{Success Criteria}
% Give critia that can be used to test if you've achieved goals, and explain the form evidence will be included

This project will be deemed a success if I complete the following tasks:
\begin{itemize}
  \item Develop a baseline search algorithm that uses the proto-voice model to output a partial reduction of a piece of music up to the chord labels. 
  \item Create an evaluation module that can take the output of the search algorithm and quantitatively evaluate its accuracy against the ground truth annotations by providing a score based on a statistical harmony model. 
    % \\ To make this concrete, an expert will manually create partial reductions, some that represent infeasible interpretations, and some that represent feasible interpretations. This module should assign expertly crafted partial reductions very high scores and infeasible partial reductions that are music theoretically unsound, low scores.   
  \item Extension: Develop one or more search algorithms that use additional heuristics to inform the search, and compare the accuracy with the baseline algorithm.
\end{itemize}

\section{Timetable}
\setlength{\extrarowheight}{.4em}
\begin{tabularx}{\textwidth}{@{}l  p{180pt} p{110pt} @{}}
  \toprule
  Time frame       & Work & Evidence   \\ 
  \midrule
  \textbf{Michaelmas} (Oct 4 to Dec 2)  & &    \\ 
  Oct 14 to Oct 24 & \textit{Oct 14}: Final proposal deadline.
                      Preparation work: familiarise myself with the dataset and the proto-voice model implementation.   
                      Work on manipulating reductions using the proto-voice parser provided by the paper. & None       \\
  Oct 24 to Nov 7 &  Dataset preparation and handling. & Plot useful metrics about the dataset using Haskell        \\
  Nov 7 to Nov 21 &  Random Search implementation &   None     \\
  Nov 21 to Dec 5 &  Evaluation Module. Continue with search implementation. & Evaluate a manually created derivation and plot results \\
  \midrule
  \textbf{Vacation} (Dec 3 to Jan 16)  & &    \\ 
  Dec 5 to Dec 11 &  Evaluate performance of random search. Begin to work on extensions & Plot results       \\
  Dec 10 to Dec 21 &  Trial different heuristics. Implement an end-to-end pipeline from input to evaluation. & None    \\
  Dec 21 to Dec 27 &  None & None    \\
  Dec 27 to Jan 10 &  Continue trialing and evaluating heuristics & \textit{Fulfill success criterion: At least one heuristic technique gives better performance than random search.}\\
  \midrule
  \textbf{Lent} (Jan 17 to  Mar 17)   &  &    \\ 
  Jan 4 to Jan 20 &  Buffer Period to help keep on track& None       \\
  Jan 20 to Feb 3 
                  &
                \textit{Feb 3}: Progress Report Deadline.
                  Write progress report and prepare presentation. 
                  Write draft \textit{Evaluation} chapter  & Progress Report (approx. 1 page)      \\

                  Feb 3 to Feb 17 & Prepare presentation. & \textit{Feb 8 -- 15}: Progress Report presentation       \\
  Feb 17 to Mar 3 & \textit{Feb 17}: How to write a Dissertation briefing. Write draft Introduction and Preparation chapters. Incorporate feedback on Evaluation chapter. &Send draft Introduction and Preparation chapter to supervisor       \\
  Mar 3 to Mar 17 & Write draft Implementation chapters. Incorporate feedback on Introduction and Preparation chapters.& Send draft Implementation chapters to Supervisor       \\
  \midrule
  \textbf{Vacation} (Mar 18 to Apr 24)  & &    \\ 
  Mar 18 to Mar 31 & Complete draft dissertation. & Send draft dissertation to supervisor       \\
  Mar 31 to April 15 & Give time for supervisor to read dissertation & None \\
  April 15 to April 25 & Incorporate feedback, make final revisions and checks of the whole dissertation & Submit dissertation and source code\\
  \midrule
  \textbf{Easter} (Apr 25 to Jun 16)  &  &    \\ 
   April 25 to May 12 & None      & None             \\ 
  \bottomrule
\end{tabularx}
% \section{Evaluation}
% Given the goal is to infer the harmonic annotations given a piece, the main metric by which this project will be evaluated will be based on how accurate the inferred annotations are. 
% This metric will be calculated using perceptually-informed chord label evaluation, comparing the inferred annotations with the ground truth.

\section{Resources}
I plan to use my own laptop for development: MacBook Pro 16-inch, M1 Max, 32GB Ram, 1TB SSD, 24-core GPU.

All code will be stored on a GitHub repository, which will guarantee protection from data loss. I will easily be able to switch to using university provided computers upon hardware/software failure.

The project will be built upon work that has been done in the DCML (Digital cognitive musicology lab) based in EPFL. The files are in their Github repository, and I have been granted permission to access their in-house datasets of score annotations, as well as software packages which are used to handle the data.
  
\section{Supervisor Information} 
Peter Harrison, head of Centre for Music and Science at Cambridge, has agreed to supervise me for this. 
We have agreed on a timetable for supervisions for this year. I am also working with Christoph Finkensiep, a PHD student at the DCML, and originator of the proto-voice model.
Professor Larry Paulson has agreed to be the representative university teaching officer.

% \bibliography{Dissertation}
% \bibliographystyle{plain}

\end{document}

% Work Packages:
% \begin{enumerate}
%   \item (14 Oct - 24 Oct) Dataset preparation
%     \begin{itemize}
%       \item Determine a representation of segments and annotations that can be used by the system and evaluation module.
%     \end{itemize}
%   \item (24 Oct - 5 Dec) Initial search pipeline
%     \begin{itemize}
%       \item Implement a basic graph search algorithm that can reduce a piece up to a single slice for each given segment
%         \begin{itemize}
%           \item The first iteration of this will reduce each segment independently based on just local information.
%         \end{itemize}
%     \end{itemize}
%   \item (5 Dec - 21 Dec) Evaluation module
%     \begin{itemize}
%       \item Iterate over the 'ground truth' dataset(contains the score and chord labels), and evaluates how well the labels are retrieved.
%     \end{itemize}
%   \item (21 Dec - 4 Jan) Progress report
%   \item (4 Jan - 25 Jan) Improved heuristic search approach
%     \begin{itemize}
%       \item Incorporate contextual information and heuristics to create an improved system to infer harmonic annotations
%       \item Ideally I will have 2-3 different systems by this point that can be evaluated against each other, each with incrementally increasing sophistication.
%     \end{itemize}
%   \item (25 Jan - 1 Feb) Evaluation  
%   \item (1 Feb - 12 March) Dissertation write up (6 weeks)
% \end{enumerate}




\end{document}
