% Template for a Computer Science Tripos Part II project dissertation
\documentclass[12pt,a4paper,twoside,openright]{report}
\usepackage[pdfborder={0 0 0}]{hyperref}    % turns references into hyperlinks
\usepackage[left=25mm, right=25mm, bottom=25mm, top=20mm]{geometry}  % adjusts page layout
\usepackage{graphicx}  % allows inclusion of PDF, PNG and JPG images
\usepackage{verbatim}
\usepackage{tabu}
\usepackage{docmute}   % only needed to allow inclusion of proposal.tex
\usepackage[utf8]{inputenc}
\usepackage{mathtools}
\usepackage{changepage}
\usepackage{url}
\usepackage{blindtext}
\usepackage{tabularx,booktabs}
\usepackage{dirtree}
\usepackage{cite}
\usepackage{float} % Prevent Latex from repositioning tables
\usepackage{amsmath}
\usepackage{graphicx}

%--- Remove hbox warning
\hfuzz=5000.002pt 
%----------------------------------------------------------------------------------------

% Formatting Commands
\newcommand{\keyword}[1]{\textbf{#1}}
\newcommand{\tabhead}[1]{\textbf{#1}}
\newcommand{\code}[1]{\texttt{#1}}
\newcommand{\file}[1]{\texttt{\bfseries#1}}
\newcommand{\option}[1]{\texttt{\itshape#1}}

%----------------------------------------------------------------------------------------
%% *****************************************************************
\newlength{\upBranch} % shift up the text  lines <<<<
\setlength{\upBranch}{0.7ex} % 

\newlength{\tolineSpace} % blank space bellow text  lines  <<<
\setlength{\tolineSpace}{1mm}% 

\usepackage{xpatch} % needed <<<<<<<<
\makeatletter

\xpatchcmd{\dirtree} % root
{\vbox{\@nameuse{DT@body@1}}}
{\raisebox{-\tolineSpace}{\vbox{\@nameuse{DT@body@1}}}}
{}{}    

\xpatchcmd{\dirtree} % below space
{\advance\dimen\z@ by-\@nameuse{DT@lastlevel@\the\DT@countiv}\relax}
{\advance\dimen\z@ by-\tolineSpace \advance\dimen\z@ by-\@nameuse{DT@lastlevel@\the\DT@countiv}\relax}
{}{}
    
\xpatchcmd{\dirtree}% shift up the text  lines
{\kern\DT@sep\box\z@\endgraf}
{\kern\DT@sep\raisebox{-\upBranch}{\box\z@}\endgraf}
{}{}    

\makeatother
%% *****************************************************************


%-----
% Language setting
\usepackage[english]{babel}
%\raggedbottom                           % try to avoid widows and orphans
\sloppy
\clubpenalty1000%
\widowpenalty1000%

\renewcommand{\baselinestretch}{1.1}    % adjust line spacing to make
                                        % more readable

\begin{document}

% Change these

\newcommand{\mcandidate}{2200D}
\newcommand{\mfullname}{Judah Daniels}
\newcommand{\mcollege}{Clare College}
\newcommand{\mtitle}{Inferring Harmony from Free Polyphony}
\newcommand{\ntitle}{Inferring Harmony from Free Polyphony}
\newcommand{\mexamination}{Computer Science Tripos -- Part II}
\newcommand{\mdate}{July, 2023}
\newcommand{\moriginator}{Christoph Finkensiep}
\newcommand{\msupervisor}{Dr Peter Harrison}
\newcommand{\mwordcount}{5434}
\newcommand{\mlinecount}{2272}
% Consent to the dissertation made available to University members
\newcommand{\mconsent}{I am content for my dissertation to be made available to the students and staff of the University.}
% For the Declaration of originality
\newcommand{\msignature}{Judah Daniels}


\bibliographystyle{acm}

%%%%%%%%%%%%%%%%%%%%%%%%%%%%%%%%%%%%%%%%%%%%%%%%%%%%%%%%%%%%%%%%%%%%%%%%
% Title
%%%%%%%%%%%%%%%%%%%%%%%%%%%%%%%%%%%%%%%%%%%%%%%%%%%%%%%%%%%%%%%%%%%%%%%%

\thispagestyle{empty}

\rightline{\LARGE \textbf{\mfullname}}

\vspace*{60mm}
\begin{center}
\Huge
\textbf{\mtitle} \\[5mm]
\mexamination \\[5mm]
\mcollege \\[5mm]
\mdate  % today's date
\end{center}

%%%%%%%%%%%%%%%%%%%%%%%%%%%%%%%%%%%%%%%%%%%%%%%%%%%%%%%%%%%%%%%%%%%%%%%%%%%%%%
% Proforma, table of contents and list of figures
%%%%%%%%%%%%%%%%%%%%%%%%%%%%%%%%%%%%%%%%%%%%%%%%%%%%%%%%%%%%%%%%%%%%%%%%%%%%%%

\pagestyle{plain}

\newpage
\newpage
\section*{Declaration of originality}

I, \mfullname{} of \mcollege, being a candidate for Part II of the Computer Science Tripos, hereby declare that this dissertation and the work described in it are my own work, unaided except as may be specified below, and that the dissertation does not contain material that has already been used to any substantial extent for a comparable purpose. \mconsent

\bigskip
\leftline{Signed \msignature}
\bigskip
\leftline{Date \today}

\chapter*{Proforma}


{\large
\begin{tabular}{ll}
Candidate Number:   & \bf \mcandidate                   \\
Project Title:      & \bf \mtitle                       \\
Examination:        & \bf \mexamination, \mdate         \\
Word Count:         & \bf \mwordcount\footnotemark[1]   \\
Code Line Count:    & \bf \mlinecount                   \\
Project Originator: & \bf \moriginator                  \\
Supervisor:         & \bf \msupervisor                  \\ 
\end{tabular}
}

\footnotetext[1]{This word count was computed
by \texttt{detex diss.tex | tr -cd '0-9A-Za-z $\tt\backslash$n' | wc -w}
}
\stepcounter{footnote}


\section*{Original Aims of the Project}
% At most 100 words



\section*{Work Completed}
% At most 100 words

All that has been completed appears in this dissertation.

\section*{Special Difficulties}
% At most 100 words

None

\newpage

\tableofcontents

\listoffigures

\newpage
\section*{Acknowledgements}



%%%%%%%%%%%%%%%%%%%%%%%%%%%%%%%%%%%%%%%%%%%%%%%%%%%%%%%%%%%%%%%%%%%%%%%
% now for the chapters

\pagestyle{headings}

\chapter{Introduction}
\textit{This dissertation explores efficient search strategies for parsing a symbolic music data using a musical grammar. We present ... which extends a recent model, addressing problems of intractability by using Heuristic search methods. We will see that that my novel heuristic search method . }

\section{Motivation}

A piece of music can be described using a sequence of chords, representing a higher level harmonic structure of a piece. There is a small, finite set of chord types, but each chord can be realised on the musical surface in a practically infinite number of ways. Given a score, we wish to infer the underlying chord types.

\par
The paper \textit{Modeling and Inferring Proto-voice Structure in Free Polyphony} describes a generative model that encodes the recursive and hierarchical dependencies between notes, giving rise to a grammar-like hierarchical system \cite{finkensiepMODELINGINFERRINGPROTOVOICE2021}. This proto-voice model can be used to reduce a piece into a hierarchical structure which encodes an understanding of the tonal/harmonic relations.
\par
Christoph Finkensiep suggests in his thesis that the proto-voice model may be an effective way to infer higher level latent entities, such as harmonies or voice leading schemata. Thus in this project I will ask the question: is this parsing model an effective way to annotate harmonies? By ‘effective’ we are referring to two things:
\begin{itemize}
  \item Accuracy: can the model successfully emulate how experts annotate harmonic progressions in musical passages? 
  \item Practicality: can the model be used to do this within a reasonable time frame?
\end{itemize}

While the original model could in theory be used to generate harmonic annotations, its exhaustive search strategy would be prohibitively time-consuming in practice for any but the shortest musical extracts; one half measure can have over 100,000 valid derivations \cite{finkensiepStructureFreePolyphony2023}. My approach will be to explore the use of heuristic search algorithms to solve this problem.

\section{Related Work}


\section{Aims}

%%%%%%%%%%%%%%%%%%%%%%%%%%%%%%%%%%%%%%%%%%%%%%%%%%%%%%%%%%%%%%%%%%%%%
% Preparation
%%%%%%%%%%%%%%%%%%%%%%%%%%%%%%%%%%%%%%%%%%%%%%%%%%%%%%%%%%%%%%%%%%%%%

\chapter{Preparation}
\textit{In this chapter, I present the work which was undertaken before the code was written. After a brief description of my starting point, I provide an exposition of the Proto-voice Model which forms the foundation of this project. Subsequently, I discuss probabilistic programming and Bayesian inference, including a probabilistic model of harmony. Finally, I describe the software engineering techniques and principles used throughout the project. }

\section{Starting Point}

\subsection{Relevant courses and experience}

\subsection{Existing codebase}

\section{The Proto-voice Model}

\subsection{Proto-voices}
TODO: Brief description of what notes are, what is a piece of music, introduce relevant musical terminology (Score, note, protovoice, repetition, neighbor/ ornament(choose one?)). Conflict of meaning with root-note (generative operation) and root note (musical terminology). What are the main assumptions/ infortion we need to know in order to understand the proto-voice model?

\par
TODO: Motivation of the model as a generative model on the note level, describing the piece as a DAG (Directed Acyclic Graph). What is a proto-voice exactly? 
\par 

The proto-voice model is characterised by 3 primitive generative operations on notes.

\begin{itemize}
  % \setlength\itemsep{1em}
 \item Repetitions 
  \item Neighbor notes 
  \item Passing notes
\end{itemize}

Operations with two notes are represented by edge replacement. 
\[p_1 \to p_2 ~~~\implies~~ p_1 \to c \to p_2 \label{edge replacement}\]

\subsection{Proto-voice Operations}

To model simulataneity of notes we introduce slices, which are multisets of pitches, representing segments of a piece where a group of notes are heard. 
\par 
Diagram showing a slice + a diagram showing a higher level slices, grouping an arpegiation.
\par 
A slice $m$ is defined as a multiset of pitches.
\par 
A transition $t = (s_l, e, s_r)$  relates two slices with a configuration of edges $e=(e_{reg}, e_{pass})$, a set of regular edges (repetition or neighbor), and a set of passing edges.
\par
Outer operations (Diagram of all three operations): 
\par
Split: \[t \to t'_l s' t'_r\]
Spread: \[t_l~s~t_r \to t'_l~s'_l~t'_m~s'_r~t'_r\]
Freeze: \[t \to t \]

\subsection{Proto-voice harmony}
How do we get from a proto-voice (partial)derivation to a harmonic inference?
\par
Explain what harmony is, and how the proto-voice model allows us to capture harmony. 


\section{Probabilistic Programming}

\subsection{Bayesian Inference}

\subsection{Probabilistic Model of Harmony}

\section{Heuristic Search Algorithms}

\section{Requirements Analysis}

\subsection{Main deliverables}

\subsection{Dependency Analysis}

\section{Software Engineering Techniques}
Justified and documented selection of suitable tools; good engineering approach.

\subsection{Development model}
Include Gantt chart.

\subsection{Languages, libraries and tools}
The chapter will also cite any new programming languages and systems which had to be learnt 

\begin{table}
  {
  \small
  \caption{Languages, libraries and tools}
  \label{Languages}
  \begin{center}
    \begin{tabularx}{.9\textwidth}{cXc}
      Tool & Purpose & License \\
      \toprule
      Haskell & Main language & ... \\
      \midrule
      GHC & Compiling and profiling to inspect time performance and memory usage  & GPL-3.0+ \\
      \midrule
      Haskell-Musicology & ... & ... \\
      \midrule
      Dimcat & ... & ... \\
      \midrule
      Python & ... & ... \\
      \midrule
      Numpy & ... & ... \\
      \midrule
      Pandas & ... & ... \\
      \midrule
      MS3 & ... & ... \\
      \midrule
      Musescore 3 & ... & ... \\
      \midrule
      Protovoice Annotation Tool & ... & ... \\
      \midrule
      Git & Version Control, Continuous Integration & ... \\
      \bottomrule
    \end{tabularx}
  \end{center}
  }
\end{table}

%%%%%%%%%%%%%%%%%%%%%%%%%%%%%%%%%%%%%%%%%%%%%%%%%%%%%%%%%%%%%%%%%%%%%
% Implementation
%%%%%%%%%%%%%%%%%%%%%%%%%%%%%%%%%%%%%%%%%%%%%%%%%%%%%%%%%%%%%%%%%%%%%

\chapter{Implementation}

\section{Repository Overview:}

\DTsetlength{0em}{1.3em}{0em}{0.7pt}{3pt}       
\setlength{\DTbaselineskip}{15pt}  %minimum size for \normalsize
\renewcommand{\DTstyle}{\ttfamily}

The following describes an overview of the project repository:
\medskip
\begin{table}[h]
  % \centering
  \caption{Repository Overview}
  \vspace{\baselineskip}
  \label{jeff}
  \begin{tabularx}{\textwidth}{l X c}
    File/Folder & Description & LOC \\
    \toprule
    \toprule
  \begin{minipage}[t]{5.3cm}
    \dirtree{%
    .1 protovoices-haskell/.
    .2 src/.
    .3 HeuristicParser.hs,~HeuristicSearch.hs \vspace{\DTbaselineskip}.
    .3 RandomChoiceSearch.hs,~RandomSampleParser.hs\vspace{2\DTbaselineskip}.
    .3 Heuristics.hs,~PBHModel.hs \vspace{2\DTbaselineskip}.
    .3 FileHandling.hs\vspace{2\DTbaselineskip}.
    .3 \dots \vspace{\DTbaselineskip}.
    .2 app/.
    .3 MainFullParse.hs\vspace{\DTbaselineskip}. 
    .2 harmonic-inference \vspace{\DTbaselineskip}.
    .2 experiments/.
    .3 preprocess.ipynb.
    .3 dcml\_params.json.
    .3 inputs/ \vspace{\DTbaselineskip}.
    .2 test/ \vspace{\DTbaselineskip}.
    }
  \end{minipage} &
  \begin{minipage}[t]{8cm}
Root directory
\vspace{2\baselineskip}\\
Core Implementation (Section x)
\vspace{2\baselineskip}\\
Baseline Implemetation (Section x)
\vspace{2\DTbaselineskip}\\
Extension Implementation (Section x) 
\vspace{2\DTbaselineskip}\\
Utilities
\vspace{5\DTbaselineskip}\\
Entry Point
\vspace{8.2\DTbaselineskip}\\
Unit Tests (Section x)

  \end{minipage} & 
  \begin{minipage}[t]{0.5cm}
    2272
    \vspace{0.1\DTbaselineskip}\\
    470\\
    \vspace{\DTbaselineskip}
    121\\
    \vspace{\DTbaselineskip}
    383\\
    \vspace{1.8\DTbaselineskip}
    188\\
    \vspace{3.7\DTbaselineskip}
    431\\
    \vspace{3\DTbaselineskip}
    115\\
    \vspace{2.5\DTbaselineskip}
    611\\
  \end{minipage}
\end{tabularx}
\end{table}

\section{Core Implementation}
\subsection{Heuristic Parser}
\section{Baseline implementation}
\section{Random Sample Parser}
\section{Random Choice Search}
\section{Extension Implementation}
\subsection{Probabilistic Model of Harmony}
\subsection{Heuristic Design}
\subsection{Heuristic Search}
\section{Testing}

%%%%%%%%%%%%%%%%%%%%%%%%%%%%%%%%%%%%%%%%%%%%%%%%%%%%%%%%%%%%%%%%%%%%%
% Evaluation
%%%%%%%%%%%%%%%%%%%%%%%%%%%%%%%%%%%%%%%%%%%%%%%%%%%%%%%%%%%%%%%%%%%%%

\chapter{Evaluation}
\textit{In this chapter, I provide qualitative and quantitative evaluations of the work completed. I then provide and interpret evidence to show that the success criteria were met.}

\textit{The main questions to answer are as follows:}
\begin{itemize}
  \item \textit{Can the proto-voice model be used to accurately infer chord labels?}
  \item \textit{Can the proto-voice model be used to practically infer chord labels?}
  \item \textit{How well my heuristic search algorithms infer chord labels?}
\end{itemize}

\section{Accuracy}
Things to note
\begin{itemize}
  \item The fact that segmentation is known ahead of time provides a great deal of information \cite{gothamWhatIfWhen2021}
  \item So we can use comparisons between the random sample from each segment algorithm and the random parse algorithm to see if the use of the grammar provides an advantage over just sampling the notes directly, without looking at relations between notes.
  \item Then we want a heuristic search algorithm that considers each option exhaustively and finds the best local option. This is too computationally expensive to be used for whole pieces. 
  \item Given there can be millions of possible next states in the search, we need to look at different strategies to avoid searching through them all. E.g just sample states. 
  \item Sensitivity Analysis for the heuristic search is useful for the evaluation. Explore how robust it is to handcrafted attacks/ different types of passages.
\end{itemize}

\section{Performance}

\section{Heuristic Search (Extension)}

\section{Success Criteria}

\section{Limitations}


%%%%%%%%%%%%%%%%%%%%%%%%%%%%%%%%%%%%%%%%%%%%%%%%%%%%%%%%%%%%%%%%%%%%%
% Conclusions
%%%%%%%%%%%%%%%%%%%%%%%%%%%%%%%%%%%%%%%%%%%%%%%%%%%%%%%%%%%%%%%%%%%%%

\chapter{Conclusions}
\textit{In this chapter, I first discuss the success achieved by the project then offer a reflection on lessons learned. Finally, I consider the directions in which there is potential for future work.}
\section{Achievements}

\section{Lessons learned}

\section{Future Work}

\cite{finkensiep_modeling_2021}


%%%%%%%%%%%%%%%%%%%%%%%%%%%%%%%%%%%%%%%%%%%%%%%%%%%%%%%%%%%%%%%%%%%%%
% the bibliography
%%%%%%%%%%%%%%%%%%%%%%%%%%%%%%%%%%%%%%%%%%%%%%%%%%%%%%%%%%%%%%%%%%%%%
\addcontentsline{toc}{chapter}{Bibliography}
\nocite{*}
% \addbibresource{Disseration.bib}
\bibliography{Dissertation}


%%%%%%%%%%%%%%%%%%%%%%%%%%%%%%%%%%%%%%%%%%%%%%%%%%%%%%%%%%%%%%%%%%%%%
% the appendices
%%%%%%%%%%%%%%%%%%%%%%%%%%%%%%%%%%%%%%%%%%%%%%%%%%%%%%%%%%%%%%%%%%%%%

\appendix

\chapter{Additional Information}

% \section{metadata.tex}
% {\scriptsize\verbatiminput{metadata.tex}}
%
% \section{main.tex}
% {\scriptsize\verbatiminput{main.tex}}
%
% \section{proposal.tex}
% {\scriptsize\verbatiminput{proposal.tex}}
%
% \chapter{Makefile}
%
% \section{makefile}\label{makefile}
% {\scriptsize\verbatiminput{makefile.txt}}
%
% \section{refs.bib}
% {\scriptsize\verbatiminput{refs.bib}}


\chapter{Project Proposal}

% Note: this file can be compiled on its own, but is also included by
% diss.tex (using the docmute.sty package to ignore the preamble)
\documentclass[12pt,a4paper,twoside]{article}
\usepackage[pdfborder={0 0 0}]{hyperref}
\usepackage[margin=25mm]{geometry}
\usepackage{graphicx}
\usepackage{parskip}
\begin{document}

\begin{center}
\Large
Computer Science Tripos -- Part II -- Project Proposal\\[4mm]
\LARGE
How to write a dissertation in \LaTeX\\[4mm]

\large
M.~Richards, St John's College

Originator: Dr M.~Richards

14 October 2011
\end{center}

\vspace{5mm}

\textbf{Project Supervisor:} Dr M.~Richards

\textbf{Director of Studies:} Dr M.~Richards

\textbf{Project Overseers:} Dr F.~H.~King  \& Dr A.~W.~Moore

% Main document

\section*{Introduction}

\emph{The problem to be addressed.}

Many students write their CST dissertations in \LaTeX\ -- and spend a
fair amount of time learning just how to do that. The purpose of this
project is to write a demonstration dissertation that provides
a starting point to show how it is done.

This core proposal document will be augmented by a separately-printed
cover sheet at the front and a resource form at the end. Additional
sheets for risk assessment and human resources may also need to be
included.

This document will elaborate much of the material that is summarised on
the additional sheets.

\section*{Starting point}

\emph{Describe existing state of the art, previous work in this area,
  libraries and databases to be used. Describe the state of any
  existing codebase that is to be built on.}

I am already able to write prose using the English language. I have an
online dictionary, etc.

\section*{Resources required}

\emph{A note of the resources required and confirmation of access.}

For this project I shall mainly use my own quad-core computer that
runs Fedora Linux. Backup will be to github and/or to an SVN
repository on an external hard disk that is dumped to writable CD/DVD
media. I have another similar computer to hand should my main machine
suddenly fail. I require no other special resources.

\section*{Work to be done}

\emph{Describe the technical work.}

The project breaks down into the following sub-projects:

\begin{enumerate}

\item The construction of a skeleton dissertation with the required
  structure. This involves writing the Makefile and making dummy
  files for the title page, the proforma, chapters 1 to 5, the
  appendices and the proposal.

\item Filling in the details required in the cover page and proforma.

\item Writing the contents of chapters 1 to 5, including examples of
  common \LaTeX\ constructs.

\item Adding a example of how to use floating figures and ``encapsulated
  PostScript'' or PDF diagrams.

\end{enumerate}

\section*{Success citeria}

\emph{Describe what you expect to be able to demonstrate at the
end of the project and how you are going to evaluate your achievement.}

The project will be a success if I have a completed dissertation with
the correct chapter titles and I have achieved my other success
criteria, which are to blah \ldots


\section*{Possible extensions}

{\em Potential further envisaged evaluation metrics or extensions.}

If I achieve my main result early I shall try the following
alternative experiment or method of evaluation \ldots


\section*{Timetable}

\emph{A workplan of perhaps ten or so two-week work-packages,
as well as milestones to be achieved along the way. Provide a
target date for each milestone.}

Planned starting date is 16/10/2011.

\begin{enumerate}

\item \textbf{Michaelmas weeks 2--4} Learn to use X. Read book Y. Read papers Z.

\item \textbf{Michaelmas weeks 5--6} Do preliminary test of Q.

\item \textbf{Michaelmas weeks 7--8} Start implementation of main task A.

\item \textbf{Michaelmas vacation} Finish A and start main task B.

\item \textbf{Lent weeks 0--2} Write progress report. Generate corpus of
  test examples. Finish task B.

\item \textbf{Lent weeks 3--5} Run main experiments and achieve working project.

\item \textbf{Lent weeks 6--8} Second main deliverable here.

\item \textbf{Easter vacation:} Extensions and writing dissertation main
  chapters.

\item \textbf{Easter term 0--2:}  Further evaluation and complete dissertation.

\item \textbf{Easter term 3:} Proof reading and then an early submission
  so as to concentrate on examination revision.

\end{enumerate}

\end{document}

\end{document}
