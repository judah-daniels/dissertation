% Template for a Computer Science Tripos Part II project dissertation
\documentclass[12pt,a4paper,twoside,openright]{report}
\usepackage[pdfborder={0 0 0}]{hyperref}    % turns references into hyperlinks
\usepackage[margin=25mm]{geometry}  % adjusts page layout
\usepackage{graphicx}  % allows inclusion of PDF, PNG and JPG images
\usepackage{verbatim}
\usepackage{docmute}   % only needed to allow inclusion of proposal.tex
\usepackage[utf8]{inputenc}
\usepackage{mathtools}
\usepackage{changepage}
\usepackage{url}
\usepackage{blindtext}
\usepackage{tabularx,booktabs}
\usepackage{dirtree}
\usepackage{cite}
\usepackage{float} % Prevent Latex from repositioning tables
\usepackage{amsmath}
\usepackage{graphicx}

%----------------------------------------------------------------------------------------

% Formatting Commands
\newcommand{\keyword}[1]{\textbf{#1}}
\newcommand{\tabhead}[1]{\textbf{#1}}
\newcommand{\code}[1]{\texttt{#1}}
\newcommand{\file}[1]{\texttt{\bfseries#1}}
\newcommand{\option}[1]{\texttt{\itshape#1}}

%----------------------------------------------------------------------------------------

% Language setting
\usepackage[english]{babel}
%\raggedbottom                           % try to avoid widows and orphans
\sloppy
\clubpenalty1000%
\widowpenalty1000%

\renewcommand{\baselinestretch}{1.1}    % adjust line spacing to make
                                        % more readable

\begin{document}

% Change these

\newcommand{\mcandidate}{2200D}
\newcommand{\mfullname}{Judah Daniels}
\newcommand{\mcollege}{Clare College}
\newcommand{\mtitle}{Inferring Harmony from Free Polyphony}
\newcommand{\ntitle}{Inferring Harmony from Free Polyphony}
\newcommand{\mexamination}{Computer Science Tripos -- Part II}
\newcommand{\mdate}{July, 2023}
\newcommand{\moriginator}{Christoph Finkensiep}
\newcommand{\msupervisor}{Dr Peter Harrison}
\newcommand{\mwordcount}{5434}
\newcommand{\mlinecount}{2272}
% Consent to the dissertation made available to University members
\newcommand{\mconsent}{I am content for my dissertation to be made available to the students and staff of the University.}
% For the Declaration of originality
\newcommand{\msignature}{Judah Daniels}


\bibliographystyle{plain}

%%%%%%%%%%%%%%%%%%%%%%%%%%%%%%%%%%%%%%%%%%%%%%%%%%%%%%%%%%%%%%%%%%%%%%%%
% Title
%%%%%%%%%%%%%%%%%%%%%%%%%%%%%%%%%%%%%%%%%%%%%%%%%%%%%%%%%%%%%%%%%%%%%%%%

\thispagestyle{empty}

\rightline{\LARGE \textbf{\mfullname}}

\vspace*{60mm}
\begin{center}
\Huge
\textbf{\mtitle} \\[5mm]
\mexamination \\[5mm]
\mcollege \\[5mm]
\mdate  % today's date
\end{center}

%%%%%%%%%%%%%%%%%%%%%%%%%%%%%%%%%%%%%%%%%%%%%%%%%%%%%%%%%%%%%%%%%%%%%%%%%%%%%%
% Proforma, table of contents and list of figures
%%%%%%%%%%%%%%%%%%%%%%%%%%%%%%%%%%%%%%%%%%%%%%%%%%%%%%%%%%%%%%%%%%%%%%%%%%%%%%

\pagestyle{plain}

\newpage
\newpage
\section*{Declaration of originality}

I, \mfullname{} of \mcollege, being a candidate for Part II of the Computer Science Tripos, hereby declare that this dissertation and the work described in it are my own work, unaided except as may be specified below, and that the dissertation does not contain material that has already been used to any substantial extent for a comparable purpose. \mconsent

\bigskip
\leftline{Signed \msignature}
\bigskip
\leftline{Date \today}

\chapter*{Proforma}

{\large
\begin{tabular}{ll}
Candidate Number:   & \bf \mcandidate                   \\
Project Title:      & \bf \mtitle                       \\
Examination:        & \bf \mexamination, \mdate         \\
Word Count:         & \bf \mwordcount\footnotemark[1]   \\
Code Line Count:    & \bf \mlinecount                   \\
Project Originator: & \bf \moriginator                  \\
Supervisor:         & \bf \msupervisor                  \\ 
\end{tabular}
}

\footnotetext[1]{This word count was computed
by \texttt{detex diss.tex | tr -cd '0-9A-Za-z $\tt\backslash$n' | wc -w}
}
\stepcounter{footnote}


\section*{Original Aims of the Project}
% At most 100 words



\section*{Work Completed}
% At most 100 words

All that has been completed appears in this dissertation.

\section*{Special Difficulties}
% At most 100 words

None

\newpage

\tableofcontents

\listoffigures

\newpage
\section*{Acknowledgements}

This document owes much to an earlier version written by Simon Moore
\cite{Moore95}.  His help, encouragement and advice was greatly 
appreciated.

%%%%%%%%%%%%%%%%%%%%%%%%%%%%%%%%%%%%%%%%%%%%%%%%%%%%%%%%%%%%%%%%%%%%%%%
% now for the chapters

\pagestyle{headings}

\chapter{Introduction}

\section{Motivation}
The introduction should explain the principal motivation for the project and show how the work fits into the broad area of surrounding computer science and give a brief survey of previous related work. \\
Clear motivation, justifying potential benefits of success.
\section{Related Work}
It should generally be unnecessary to quote at length from technical papers or textbooks.
If a simple bibliographic reference is insufficient, consign any lengthy quotation to an appendix.
\section{Aims}

%%%%%%%%%%%%%%%%%%%%%%%%%%%%%%%%%%%%%%%%%%%%%%%%%%%%%%%%%%%%%%%%%%%%%
% Preparation
%%%%%%%%%%%%%%%%%%%%%%%%%%%%%%%%%%%%%%%%%%%%%%%%%%%%%%%%%%%%%%%%%%%%%

\chapter{Preparation}
\paragraph{Description: }
Principally, this chapter should describe the work which was undertaken before code was written, hardware built or theories worked on. 
It should show how the project proposal was further refined and clarified, so that the implementation stage could go smoothly rather than by trial and error.
\par
Throughout this chapter and indeed the whole dissertation, it is essential to demonstrate that a proper professional approach was employed.
\par

\section{Starting Point}
It is essential to declare the starting point. 
This states any existing codebase or materials that your project builds on. 
The text here can commonly be identical to the text in your proposal, but it may enlarge on it or report variations.
For instance, the true starting point may have turned out to be different from that declared in the proposal and such discrepancies must be explained.

\subsection{Relevant courses and experience}

\subsection{Existing codebase}

\section{The Protovoice Model}
Mention complicated theories or algorithms which required understanding. \\ 
Clear presentation of challenging background material covering a range of computer science topics beyond Part IB.

\section{Heuristic Search Algorithms}

\section{Data Processing}

\section{Requirements Analysis}
Underlining the professional approach, this chapter will very likely include a section headed "Requirements Analysis" and refer to appropriate software engineering techniques used in the dissertation. \\
Good or excellent requirements analysis; 
\par 
Main deliverables with a risk analysis. 
Implementation modules with a dependency analysis.

\section{Software Engineering Techniques}
justified and documented selection of suitable tools; good engineering approach.

\subsection{Development model}
Include Gantt chart.
\subsection{Languages, libraries, tools}
The chapter will also cite any new programming languages and systems which had to be learnt \\
Table: 
\par 
stack  \\
haskell \\ 
python \\
numpy \\ 
pandas \\ 
ms3 \\
Haskell-Musicology \\ 
protovoice annotation tool \\
memory management - ghc profiling \\
\par
The introduction should explain the principal motivation for the project and show how the work fits into the broad area of surrounding computer science and give a brief survey of previous related work.
It should generally be unnecessary to quote at length from technical papers or textbooks.
If a simple bibliographic reference is insufficient, consign any lengthy quotation to an appendix.

\paragraph{Mark Scheme: }
Good or excellent requirements analysis; justified and documented selection of suitable tools; good engineering approach.
Clear presentation of challenging background material covering a range of computer science topics beyond Part IB.

%%%%%%%%%%%%%%%%%%%%%%%%%%%%%%%%%%%%%%%%%%%%%%%%%%%%%%%%%%%%%%%%%%%%%
% Implementation
%%%%%%%%%%%%%%%%%%%%%%%%%%%%%%%%%%%%%%%%%%%%%%%%%%%%%%%%%%%%%%%%%%%%%

\chapter{Implementation}

\paragraph{Description: }
This chapter should describe what was actually produced: the programs which were written, the hardware which was built or the theory which was developed. Any design strategies that looked ahead to the testing stage should be described in order to demonstrate a professional approach was taken.
\par
Descriptions of programs may include fragments of high-level code but large chunks of code are usually best left to appendices or omitted altogether. Analogous advice applies to circuit diagrams or detailed steps in a machine-checked proof.
\par
The implementation chapter should include a section labelled "Repository Overview". The repository overview should be around one page in length and should describe the high-level structure of the source code found in your source code repository. It should describe whether the code was written from scratch or if it built on an existing project or tutorial. Making effective use of powerful tools and pre-existing code is often laudable, and will count to your credit if properly reported. Nevertheless, as in the rest of the dissertation, it is essential to draw attention to the parts of the work which are not your own. 
\par
It should not be necessary to give a day-by-day account of the progress of the work but major milestones may sometimes be highlighted with advantage.

\paragraph{Mark Scheme: }
Contribution to the field.
Application of extra-curricular reading and original interpretation of previous work from academia or industry.
Challenging goals and substantial deliverables with excellent selection and application of appropriate mathematical, scientific and/or engineering techniques.
Clear and justified repository overview.
At most minor faults in execution or understanding.

\section{Repository Overview:}
The following describes the protovoices-haskell repository, and where my code contribution will lie:
\par
\medskip
\dirtree{%
.1 protovoices-haskell.
.2 app.
.3 MainHeuristicSearch.hs <- My code.
.3 ....
.2 src.
.3 \textbf{Heuristics} <- My code.  
.4 …  <- My code.
.3 ....
.2 test.
.2 testdata.
.2 ....
}
%%%%%%%%%%%%%%%%%%%%%%%%%%%%%%%%%%%%%%%%%%%%%%%%%%%%%%%%%%%%%%%%%%%%%
% Evaluation
%%%%%%%%%%%%%%%%%%%%%%%%%%%%%%%%%%%%%%%%%%%%%%%%%%%%%%%%%%%%%%%%%%%%%

\chapter{Evaluation}

\paragraph{Description:}
This is where Assessors will be looking for signs of success and for evidence of thorough and systematic evaluation.
Sample output, tables of timings and photographs of workstation screens, oscilloscope traces or circuit boards may be included.
Care should be employed to take a professional approach throughout.
For example, a graph that does not indicate confidence intervals will generally leave a professional scientist with a negative impression. 
As with code, voluminous examples of sample output are usually best left to appendices or omitted altogether.
\par
There are some obvious questions which this chapter will address. How many of the original goals were achieved? Were they proved to have been achieved? Did the program, hardware, or theory really work?
\par
Assessors are well aware that large programs will very likely include some residual bugs. It should always be possible to demonstrate that a program works in simple cases and it is instructive to demonstrate how close it is to working in a really ambitious case.

\paragraph{Mark Scheme (Evaluation and Conclusions): }
Conclusions provide an effective summary of work completed along with good future work.
Clearly presented argument demonstrating success criteria met.
Good or excellent evidence of critical thought and interpretation of the results which substantiate any claims of success, improvements or novelty.

%%%%%%%%%%%%%%%%%%%%%%%%%%%%%%%%%%%%%%%%%%%%%%%%%%%%%%%%%%%%%%%%%%%%%
% Conclusions
%%%%%%%%%%%%%%%%%%%%%%%%%%%%%%%%%%%%%%%%%%%%%%%%%%%%%%%%%%%%%%%%%%%%%

\chapter{Conclusions}
This chapter is likely to be very short and it may well refer back to the Introduction. 
\section{Achievements}

\section{Future Work}

\section{Lessons learned}
Personal reflection on the lessons learned.
It might offer a reflection on the lessons learned and explain how you would have planned the project if starting again with the benefit of hindsight.

%%%%%%%%%%%%%%%%%%%%%%%%%%%%%%%%%%%%%%%%%%%%%%%%%%%%%%%%%%%%%%%%%%%%%
% the bibliography
%%%%%%%%%%%%%%%%%%%%%%%%%%%%%%%%%%%%%%%%%%%%%%%%%%%%%%%%%%%%%%%%%%%%%

\addcontentsline{toc}{chapter}{Bibliography}
\bibliography{refs}

%%%%%%%%%%%%%%%%%%%%%%%%%%%%%%%%%%%%%%%%%%%%%%%%%%%%%%%%%%%%%%%%%%%%%
% the appendices
%%%%%%%%%%%%%%%%%%%%%%%%%%%%%%%%%%%%%%%%%%%%%%%%%%%%%%%%%%%%%%%%%%%%%

\appendix

\chapter{Latex source}

\section{metadata.tex}
{\scriptsize\verbatiminput{metadata.tex}}

\section{main.tex}
{\scriptsize\verbatiminput{main.tex}}

\section{proposal.tex}
{\scriptsize\verbatiminput{proposal.tex}}

\chapter{Makefile}

\section{makefile}\label{makefile}
{\scriptsize\verbatiminput{makefile.txt}}

\section{refs.bib}
{\scriptsize\verbatiminput{refs.bib}}


\chapter{Project Proposal}

% Note: this file can be compiled on its own, but is also included by
% diss.tex (using the docmute.sty package to ignore the preamble)
\documentclass[12pt,a4paper,twoside]{article}
\usepackage[pdfborder={0 0 0}]{hyperref}
\usepackage[margin=25mm]{geometry}
\usepackage{graphicx}
\usepackage{parskip}
\begin{document}

\begin{center}
\Large
Computer Science Tripos -- Part II -- Project Proposal\\[4mm]
\LARGE
How to write a dissertation in \LaTeX\\[4mm]

\large
M.~Richards, St John's College

Originator: Dr M.~Richards

14 October 2011
\end{center}

\vspace{5mm}

\textbf{Project Supervisor:} Dr M.~Richards

\textbf{Director of Studies:} Dr M.~Richards

\textbf{Project Overseers:} Dr F.~H.~King  \& Dr A.~W.~Moore

% Main document

\section*{Introduction}

\emph{The problem to be addressed.}

Many students write their CST dissertations in \LaTeX\ -- and spend a
fair amount of time learning just how to do that. The purpose of this
project is to write a demonstration dissertation that provides
a starting point to show how it is done.

This core proposal document will be augmented by a separately-printed
cover sheet at the front and a resource form at the end. Additional
sheets for risk assessment and human resources may also need to be
included.

This document will elaborate much of the material that is summarised on
the additional sheets.

\section*{Starting point}

\emph{Describe existing state of the art, previous work in this area,
  libraries and databases to be used. Describe the state of any
  existing codebase that is to be built on.}

I am already able to write prose using the English language. I have an
online dictionary, etc.

\section*{Resources required}

\emph{A note of the resources required and confirmation of access.}

For this project I shall mainly use my own quad-core computer that
runs Fedora Linux. Backup will be to github and/or to an SVN
repository on an external hard disk that is dumped to writable CD/DVD
media. I have another similar computer to hand should my main machine
suddenly fail. I require no other special resources.

\section*{Work to be done}

\emph{Describe the technical work.}

The project breaks down into the following sub-projects:

\begin{enumerate}

\item The construction of a skeleton dissertation with the required
  structure. This involves writing the Makefile and making dummy
  files for the title page, the proforma, chapters 1 to 5, the
  appendices and the proposal.

\item Filling in the details required in the cover page and proforma.

\item Writing the contents of chapters 1 to 5, including examples of
  common \LaTeX\ constructs.

\item Adding a example of how to use floating figures and ``encapsulated
  PostScript'' or PDF diagrams.

\end{enumerate}

\section*{Success citeria}

\emph{Describe what you expect to be able to demonstrate at the
end of the project and how you are going to evaluate your achievement.}

The project will be a success if I have a completed dissertation with
the correct chapter titles and I have achieved my other success
criteria, which are to blah \ldots


\section*{Possible extensions}

{\em Potential further envisaged evaluation metrics or extensions.}

If I achieve my main result early I shall try the following
alternative experiment or method of evaluation \ldots


\section*{Timetable}

\emph{A workplan of perhaps ten or so two-week work-packages,
as well as milestones to be achieved along the way. Provide a
target date for each milestone.}

Planned starting date is 16/10/2011.

\begin{enumerate}

\item \textbf{Michaelmas weeks 2--4} Learn to use X. Read book Y. Read papers Z.

\item \textbf{Michaelmas weeks 5--6} Do preliminary test of Q.

\item \textbf{Michaelmas weeks 7--8} Start implementation of main task A.

\item \textbf{Michaelmas vacation} Finish A and start main task B.

\item \textbf{Lent weeks 0--2} Write progress report. Generate corpus of
  test examples. Finish task B.

\item \textbf{Lent weeks 3--5} Run main experiments and achieve working project.

\item \textbf{Lent weeks 6--8} Second main deliverable here.

\item \textbf{Easter vacation:} Extensions and writing dissertation main
  chapters.

\item \textbf{Easter term 0--2:}  Further evaluation and complete dissertation.

\item \textbf{Easter term 3:} Proof reading and then an early submission
  so as to concentrate on examination revision.

\end{enumerate}

\end{document}

\end{document}
