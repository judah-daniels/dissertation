\documentclass{article}

% Packages
\usepackage{tabularx,booktabs}
\usepackage{dirtree}
\usepackage{cite}
\usepackage{float} % Prevent Latex from repositioning tables
\usepackage{amsmath}
\usepackage{graphicx}
\usepackage[colorlinks=true, allcolors=blue]{hyperref}

%----------------------------------------------------------------------------------------

% Formatting Commands
\newcommand{\keyword}[1]{\textbf{#1}}
\newcommand{\tabhead}[1]{\textbf{#1}}
\newcommand{\code}[1]{\texttt{#1}}
\newcommand{\file}[1]{\texttt{\bfseries#1}}
\newcommand{\option}[1]{\texttt{\itshape#1}}

%----------------------------------------------------------------------------------------

% Language setting
\usepackage[english]{babel}

% Set page size and margins
% Replace `letterpaper' with `a4paper' for UK/EU standard size
\usepackage[a4paper,top=2cm,bottom=2cm,left=3cm,right=3cm,marginparwidth=1.75cm]{geometry}

\begin{document}

\title{Inferring Harmony from Free Polyphony}
\author{Judah Daniels}
\date{\parbox{\linewidth}{\centering%
  \today\endgraf\bigskip
  DOS: Prof. Larry Paulson\endgraf\medskip
  Crsid: jasd6 \endgraf
  College: Clare College}}
\maketitle


\section{Abstract}
% Further Explanation of the background and the objectives.
A piece of music can be described using a sequence of chords, representing a higher level harmonic structure of a piece. There is a small, finite set of chord types, but each chord can be realised on the musical surface in a practically infinite number of ways. Given a score, we wish to infer the underlying chord types. 
\par
The paper \textit{Modeling and Inferring Proto-voice Structure in Free Polyphony} describes a generative model that encodes the recursive and hierarchical dependencies between notes, giving rise to a grammar-like hierarchical system \cite{finkensiep_modeling_2021}. This proto-voice model can be used to reduce a piece into a hierarchical structure which encodes an understanding of the tonal/harmonic relations of a piece. 
%A reduction is represented by a sequence of slices, where each slice is a set of pitches with a pointer to a starting and ending surface position
\par
Christoph Finkensiep suggests in his paper that the proto-voice model may be an effective way to infer higher level latent entities, such as harmonies or voice leading schemata. Thus in this project I will ask the question: is this parsing model an effective way to annotate harmonies? By `effective' we are referring to two things: \begin{itemize}
  \item Accuracy: can the model successfully emulate how experts annotate harmonic progressions in musical passages? 
  \item Practicality: can the model be used to do this within a reasonable time frame?
\end{itemize}
% Harmonic annotations segment a piece, marking each segment with a chord label. The goal is then to obtain these labels from the score given the segmentation.
While the original model could in theory be used to generate harmonic annotations, its exhaustive search strategy would be prohibitively time-consuming in practice for any but the shortest musical extracts; one half measure can have over 100,000 valid derivations \cite{finkensiep_modeling_2021}. My approach will be to explore the use of heuristic search algorithms to solve this problem.
% My project will extend the use of this model into field of harmonic analysis by inferring harmonic annotations for a piece. 
% \par
% As the protovoice model is generative, this harmonic analysis will consist of applying the inverse of the generative operations in order to find a reduction. The search-space for this model is intractable; one half measure can have over 100,000 valid derivations\cite{finkensiep_modeling_2021}.
% The starting point is a parser provided with the paper which exhaustively searches for a solution, leading to an exponential blowup. My approach will be to use heuristic search algorithms to find a partial derivation which is sufficient to infer the underlying harmony.
% the goal is basically 
% just to find a reduction of each segment to a single slice, where 
% ideally (a) this slice reflects the chord label and (b) the reduction 
% itself is a "plausible" derivation (

%The general approach would be based on (heuristic) search. The search 
% space would be partial reductions, the starting point is the (unreduced 
% and segmented) surface, and the goal would be any reduction of the 
% segments to single slices. The connections between different states are 
% the reduction steps defined by the PV model.

% My project will involve development of a parser(s) that can parse pieces of music using the grammar described in the paper, and formulate a method to search within the space of possible parse trees for a suitable structural interpretation of a piece of music. This will be used to intelligently infer chord annotations given an arbitrary score.

% The dissertation includes code for a random parser which generates a derivation of the piece according to the protovoice model, by applying the generative process backwards, randomly choosing operations. 
%
% The specific problem of inferring harmonic annotations is not addressed in the dissertation, so theres is an open question of how one can go from a parse tree to a set of chord labels.
%

\section{Substance and Structure}
% Give details of specific goals to be achieved
% Give precise characterisations of the methods that will be used
% Data structures and algorithms
% Key concepts, major work items, their relations and relative importance
% Specify what it means for the project to be a success.
% Identify the main sub tasks
% Outline the main algorithms or techniques to be adopted in completed them
\subsection{Core: Search}
The core of this project is essentially a search problem characterised as follows:
\begin{itemize}
  \item The state space $S$ is the set of all possible partial reductions of a piece along with each reduction step that has been done so far. 
  \item We have an initial state $s_o \in S$, which is the empty reduction, corresponding to the unreduced surface of the piece. The score is represented as a sequence of slices grouping notes that sound simultaneously. We are also given the segmentation of the original chord labels that we wish to retrieve.
  \item We have a set of actions, $A$ modelled by a function $action: A \times S \to S$. These actions correspond to a single reduction step.
    \begin{itemize}
      \item The reduction steps are the inverses of the operations defined by the generative proto-voice model.
    \end{itemize}
  \item Finally we have a goal test, $goal: S \to \{true,false\}$ which is true iff the partial reduction $s$ has exactly one slice per segment of the input.
    \begin {itemize}
  \item This means the partial reduction $s$ contains a sequence of slices which start and end positions corresponding to the segmentation of the piece.
    \end {itemize}
  \item At the first stage, this will be implemented using a random graph search algorithm, picking each action randomly, according to precomputed distributions.
\end{itemize}
\par

\subsection{Core: Evaluation}

The second core task is to create an evaluation module that iterates over the test dataset, and evaluates the partial reduction computed by the search algorithm above. This will be done by comparing the outputs to ground truth annotations from the Annotated Beethoven Corpus.
\par
In order to do this I will make use of the statistical harmony model from Finkensiep's thesis, \textit{The Structure of Free Polyphony} \cite{finkensiep_structure_2022}. This model provides a way of mapping between the slices that the algorithm generates and the chords in the ground truth. This can be used to empirically measure how closely the slices match the expert annotations.   
\par

% from above by measuring the likelihood that the chord label produced the slice notes.

\subsection{Extension}
Once the base search implementation and evaluation module have been completed, the search problem will be tackled by heuristic search methods, with different heuristics to be trialled and evaluated against each other. The heuristics will make use of the chord profiles from Finkensiep's statistical harmony model discussed above. These profiles relate note choices to the underlying harmony. Hence the heuristics may include:
\begin{itemize}
  \item How the chord types relate to the pitches used.
  \item How the chord types relate which notes are used as ornamentation, and the degree of ornamentation.
  \item Contextual information about neighboring slices
\end{itemize}

\subsection{Overview}
\noindent
The main work packages are as follows:
\par
\medskip
\keyword{Preliminary Reading} -- Familiarise myself with the proto-voice model, and read up on similar models and their implementations. Study heuristic search algorithms.
\par
\medskip
\keyword{Dataset Preparation} -- Pre-process the Annotated Beethoven Corpus into a suitable representation for my algorithm.
\par
\medskip
\keyword{Basic Search} -- Implement a basic random search algorithm that takes in surface and segmentations, and outputting the sequence of slices matching the segmentations.
\par
\medskip
\keyword{Evaluation Module} -- Implement an evaluation module to evaluate the output from the search algorithm.
% Chord Profiles for chord tones and ornaments
% Existing set of probabilities: refer to the thesis.
\par
\medskip
\keyword{End-to-end pipeline} -- Implement a full pipeline from the data to the evaluation that can be used to compare different reductions.
\par
\medskip
\keyword{Heuristic Design} -- Extension -- Trial different heuristics and evaluate their performance against each other.
\par
\medskip
\keyword{Dissertation} -- I intend to work on the dissertation throughout the duration of the project. I will then focus on completing and polishing the project upon completion.

\section{Starting Point}
% Record any signifcant bodies of coe that will form a basis for your project that already exists
% Describe state of existing software
The following describes existing code and languages that will be used for this project:
\par
\bigskip
\keyword{Haskell} -- I will be using Haskell for this project as it is used in the proto-voice implementation. It must be noted that my experience with Haskell is limited, as I was first introduced to it via an internship this summer (July to August 2022).
\par
\bigskip
\keyword{Python} -- Python will be used for data handling. I have experience coding in Python.
\par
\bigskip
\keyword{Prior Research} - Over the summer I have been reading the literature on computational models of music, as well as various parsing algorithms such as semi-ring parsing \cite{goodman_semiring_1999}, and the CYK algorithm, which is used in the implementation of the proto-voice model.
\par
\bigskip
\keyword{Protovoices-Haskell} -- The paper \textit{Modeling and Inferring Proto-Voice Structure in Free Polyphony}\cite{finkensiep_modeling_2021} includes an implementation of the proto-voice model in Haskell. A fork of this repository will form the basis of my project.
This repository includes as parsing module which will be used to perform the actions in the search space of partial reductions. There is module that can exhaustively enumerate reductions of a piece, but this is infeasible in practice due to the blowup of the derivation forest.
\par
\bigskip
\keyword{MS3} -- This is a library for parsing MuseScore Files and manipulating labels\cite{johannes_ms3_2021}, which I will use as part of the data processing pipeline.
\par
\bigskip
\keyword{ABC} -- The \textit{Annotated Beethoven Corpus}\cite{neuwirth_annotated_2018} contains analyses of all Beethoven string quartets composed between 1800 and 1826), encoded in a human and machine readable format. This will be used as a dataset for this project. 
\par
%
% The following describes the protovoices-haskell repository, and where my code contribution will lie:
% \par
% \medskip
% \dirtree{%
% .1 protovoices-haskell.
% .2 app.
% .3 MainExamples.hs.
% .3 MainISMIR.hs.
% .3 MainLearning.hs.
% .3 MainHeuristicSearch.hs <- My code.
% .3 ....
% .2 src.
% .3 \textbf{Heuristics} <- My code.  
% .4 …  <- My code.
% .3 ....
% .2 test.
% .2 testdata.
% .2 ....
% }
\section{Success Criteria}
% Give critia that can be used to test if you've achieved goals, and explain the form evidence will be included

This project will be deemed a success if I complete the following tasks:
\begin{itemize}
  \item Develop a baseline search algorithm that uses the proto-voice model to output a partial reduction of a piece of music up to the chord labels. 
  \item Create an evaluation module that can take the output of the search algorithm and quantitatively evaluate its accuracy against the ground truth annotations by providing a score based on a statistical harmony model. 
    % \\ To make this concrete, an expert will manually create partial reductions, some that represent infeasible interpretations, and some that represent feasible interpretations. This module should assign expertly crafted partial reductions very high scores and infeasible partial reductions that are music theoretically unsound, low scores.   
  \item Extension: Develop one or more search algorithms that use additional heuristics to inform the search, and compare the accuracy with the baseline algorithm.
\end{itemize}

\section{Timetable}
\setlength{\extrarowheight}{.4em}
\begin{tabularx}{\textwidth}{@{}l  p{180pt} p{110pt} @{}}
  \toprule
  Time frame       & Work & Evidence   \\ 
  \midrule
  \textbf{Michaelmas} (Oct 4 to Dec 2)  & &    \\ 
  Oct 14 to Oct 24 & \textit{Oct 14}: Final proposal deadline.
                      Preparation work: familiarise myself with the dataset and the proto-voice model implementation.   
                      Work on manipulating reductions using the proto-voice parser provided by the paper. & None       \\
  Oct 24 to Nov 7 &  Dataset preparation and handling. & Plot useful metrics about the dataset using Haskell        \\
  Nov 7 to Nov 21 &  Random Search implementation &   None     \\
  Nov 21 to Dec 5 &  Evaluation Module. Continue with search implementation. & Evaluate a manually created derivation and plot results \\
  \midrule
  \textbf{Vacation} (Dec 3 to Jan 16)  & &    \\ 
  Dec 5 to Dec 11 &  Evaluate performance of random search. Begin to work on extensions & Plot results       \\
  Dec 10 to Dec 21 &  Trial different heuristics. Implement an end-to-end pipeline from input to evaluation. & None    \\
  Dec 21 to Dec 27 &  None & None    \\
  Dec 27 to Jan 10 &  Continue trialing and evaluating heuristics & \textit{Fulfill success criterion: At least one heuristic technique gives better performance than random search.}\\
  \midrule
  \textbf{Lent} (Jan 17 to  Mar 17)   &  &    \\ 
  Jan 4 to Jan 20 &  Buffer Period to help keep on track& None       \\
  Jan 20 to Feb 3 
                  &
                \textit{Feb 3}: Progress Report Deadline.
                  Write progress report and prepare presentation. 
                  Write draft \textit{Evaluation} chapter  & Progress Report (approx. 1 page)      \\

                  Feb 3 to Feb 17 & Prepare presentation. & \textit{Feb 8 -- 15}: Progress Report presentation       \\
  Feb 17 to Mar 3 & \textit{Feb 17}: How to write a Dissertation briefing. Write draft Introduction and Preparation chapters. Incorporate feedback on Evaluation chapter. &Send draft Introduction and Preparation chapter to supervisor       \\
  Mar 3 to Mar 17 & Write draft Implementation chapters. Incorporate feedback on Introduction and Preparation chapters.& Send draft Implementation chapters to Supervisor       \\
  \midrule
  \textbf{Vacation} (Mar 18 to Apr 24)  & &    \\ 
  Mar 18 to Mar 31 & Complete draft dissertation. & Send draft dissertation to supervisor       \\
  Mar 31 to April 15 & Give time for supervisor to read dissertation & None \\
  April 15 to April 25 & Incorporate feedback, make final revisions and checks of the whole dissertation & Submit dissertation and source code\\
  \midrule
  \textbf{Easter} (Apr 25 to Jun 16)  &  &    \\ 
   April 25 to May 12 & None      & None             \\ 
  \bottomrule
\end{tabularx}
% \section{Evaluation}
% Given the goal is to infer the harmonic annotations given a piece, the main metric by which this project will be evaluated will be based on how accurate the inferred annotations are. 
% This metric will be calculated using perceptually-informed chord label evaluation, comparing the inferred annotations with the ground truth.

\section{Resources}
I plan to use my own laptop for development: MacBook Pro 16-inch, M1 Max, 32GB Ram, 1TB SSD, 24-core GPU.

All code will be stored on a GitHub repository, which will guarantee protection from data loss. I will easily be able to switch to using university provided computers upon hardware/software failure.

The project will be built upon work that has been done in the DCML (Digital cognitive musicology lab) based in EPFL. The files are in their Github repository, and I have been granted permission to access their in-house datasets of score annotations, as well as software packages which are used to handle the data.
  
\section{Supervisor Information} 
Peter Harrison, head of Centre for Music and Science at Cambridge, has agreed to supervise me for this. 
We have agreed on a timetable for supervisions for this year. I am also working with Christoph Finkensiep, a PHD student at the DCML, and originator of the proto-voice model.
Professor Larry Paulson has agreed to be the representative university teaching officer.

% \bibliography{Dissertation}
% \bibliographystyle{plain}

\end{document}

% Work Packages:
% \begin{enumerate}
%   \item (14 Oct - 24 Oct) Dataset preparation
%     \begin{itemize}
%       \item Determine a representation of segments and annotations that can be used by the system and evaluation module.
%     \end{itemize}
%   \item (24 Oct - 5 Dec) Initial search pipeline
%     \begin{itemize}
%       \item Implement a basic graph search algorithm that can reduce a piece up to a single slice for each given segment
%         \begin{itemize}
%           \item The first iteration of this will reduce each segment independently based on just local information.
%         \end{itemize}
%     \end{itemize}
%   \item (5 Dec - 21 Dec) Evaluation module
%     \begin{itemize}
%       \item Iterate over the 'ground truth' dataset(contains the score and chord labels), and evaluates how well the labels are retrieved.
%     \end{itemize}
%   \item (21 Dec - 4 Jan) Progress report
%   \item (4 Jan - 25 Jan) Improved heuristic search approach
%     \begin{itemize}
%       \item Incorporate contextual information and heuristics to create an improved system to infer harmonic annotations
%       \item Ideally I will have 2-3 different systems by this point that can be evaluated against each other, each with incrementally increasing sophistication.
%     \end{itemize}
%   \item (25 Jan - 1 Feb) Evaluation  
%   \item (1 Feb - 12 March) Dissertation write up (6 weeks)
% \end{enumerate}


